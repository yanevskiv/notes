\section{Uvod}
\textbf{\v{S}ta je semafor?}\\
\textbf{Semafori} (eng. Semaphore) su jedan od mehanizama koji omogu\'{c}avaju sinhronizaciju ili medjusobno isklju\v{c}ivanje izmedju vi\v{s}e niti. Semafor ima svoj internu vrednost (\texttt{int value}), red blokiranih niti na semaforu (\texttt{queue<Thread*>}), i podr\v{z}ava slede\'{c}e dve operacije:
\begin{itemize}
    \item \texttt{void wait(s)} - Ukoliko je vrednost semafora nula, nit se dodaje u red blokiranih na semaforu. Ukoliko je vrednost semafora ve\'{c}a od nula, vrednost semafora se dekrementira za 1.
    \item \texttt{void signal(s)} - Ukoliko postoje zablokirane niti u redu blokiranih, jedna nit iz reda blokiranih se deblokira. Ukoliko je red zablokiranih prazan, vrednost semafora se inkrementira za 1.
\end{itemize}
Vrednost ovako definisanog semafora ne mo\v{z}e da ode ispod nule. Takodje, u redu zablokiranih ne mogu da postoje niti ukoliko je vrednost semafora ve\'{c}a od nule.
\\\\
\textbf{Po\v{s}ten semafor} je onaj koji pozivom operacije \texttt{signal(s)} deblokira niti po FIFO redosledu (nit koja je ranije pozvala \texttt{wait(s)} \'{c}e se ranije deblokirati).\\
\textbf{Nepo\v{s}ten semafor} je onaj koji pozivom operacije \texttt{signal(s)} deblokira niti po nasumi\v{c}nom redosledu.\\\\
U ovoj skripti i na KDP-u se podrazumevaju \textbf{nepo\v{s}teni semafori}, tako da je za implementaciju FIFO redosleda neophodno npr. koristiti tehniku privatnih semafora sa \texttt{queue<int>}.
\\\\
Primer kori\v{s}\'{c}enja POSIX semafora:
\begin{lstlisting}
#include <semaphore.h>
sem_t mutex;
sem_init(&mutex, 0, 1);
...
sem_wait(&mutex) // wait 
/* CRITICAL SECTION */
sem_post(&mutex); // sginal
...
sem_destroy(&mutex);
\end{lstlisting}