\clearpage
\section{Mutual exclusion}
\textbf{\large Zadatak} Proces je pokrenuo dve niti, gde prva nit popunjava deljeni niz znakova znakom 'a', dok druga nit popunjava niz znakova znakom 'b'. Ove dve niti remete rad jedna drugoj, time \v{s}to se utrkuju (race condition) i "gaze" podatke druge niti svojim podacima. Koriste\'{c}i semafore, potrebno je za\v{s}titi niz znakova tako da jedna nit ne mo\v{z}e da zapo\v{c}ne upis dok druga nit ne potpuno zavr\v{s}i.
\\\\
\textbf{\v{S}ta je mutex?}
\\
Mutex je resurs koji samo jedna nit mo\v{z}e da "dr\v{z}i" u nekom trenutku. Kada neka nit "uzme" mutex ("zaklju\v{c}a" ga) sve ostale niti koje poku\v{s}aju isto \'{c}e se blokirati dok nit koja dr\v{z}i mutex ne vrati ("otklju\v{c}a") isti. Mutex slu\v{z}i za implementaciju \textbf{kriti\v{c}ne sekcije} u kodu. Kriti\v{c}na sekcija je deo koda koji isklju\v{c}ivo jedna nit mo\v{z}e da izvr\v{s}ava u nekom trenutku. Primer kori\v{s}\'{c}enja mutex-a pomo\'{c}u POSIX niti:
\begin{lstlisting}
pthread_mutex_t mutex;
pthread_mutex_init(&mutex);
...
pthread_mutex_lock(&mutex);
// CRITICAL SECTION
pthread_mutex_unlock(&mutex);
...
pthread_mutex_destroy(&mutex);
\end{lstlisting}
Mutex se lako implementira pomo\'{c}u binarnog semafora (semafori nisu jedini na\v{c}in da se napravi mutex!). Vrednost semafora u svrsi mutex-a se inicijalizira na vrednost \texttt{s=1}, a potom \texttt{wait(s)} ozna\v{c}ava "zauze\'{c}e" mutex-a, dok \texttt{signal(s)} ozna\v{c}ava "oslobadjanje" istog.  npr. pomo\'{c}u POSIX semafora:
\begin{lstlisting}
sem_t mutex;
sem_init(&mutex, 0, 1); // init semaphore with value 1
...
sem_wait(&mutex); // lock
// CRITICAL SECTION
sem_post(&mutex); // unlock
...
sem_destroy(&mutex);
\end{lstlisting}
\textbf{\large Re\v{s}enje}
\begin{lstlisting}
#include <thread.h>
#include <sem.h>
Sem mutex = 1;
char buffer[BUFSIZ];

void a() {
    while (true) {
        wait(mutex);
		memset(buffer, 'a', BUFSIZ);
		signal(mutex);
        usleep(50);
    }
}

void b() {
    while (true) {
        wait(mutex);
		memset(buffer, 'b', BUFSIZ)
		signal(mutex);
        usleep(50);
    }
}

int main() {
    Thread ta = createThread(a);
    Thread tb = createThread(b);
    join(ta);
    join(tb);
    return 0;
}
\end{lstlisting}