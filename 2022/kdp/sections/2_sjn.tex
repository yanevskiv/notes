\clearpage
\section{\latin Shortest job next}
\textbf{\large Zadatak} Proces je pokrenuo $N$ niti koje treba pristupe kriti\v{c}noj sekciji po prioritetnom redosledu. Naime, ukoliko postoje vi\v{s}e niti koje istovremeno \v{z}ele da udju u kriti\v{c}nu sekciju, prva ulazi ona koja ima najmanje posla. Potrebno je pomo\'{c}u semafora implementirati metodu \texttt{sjn\_wait(id, prio)} (ni\v{z}a vrednost argumenta \texttt{prio} zna\v{c}i vi\v{s}i prioritet) koja se poziva pre ulaska u kriti\v{c}nu sekciju, i metodu \texttt{sjn\_signal()} koja se poziva nakon izlaska iz kriti\v{c}ne sekcije.
\\\\
\textbf{\v{S}ta je tehnika privatnih semafora?}\\
Privatni semafor neke niti je semafor inicijalizovan na vrednost 0 koji nit mo\v{z}e da koristi da se ve\v{s}ta\v{c}ki zablokira i sa\v{c}eka dok je neka druga nit ne deblokira. Pre blokiranja na privatnom semaforu neophodno je osloboditi \texttt{mutex} ukoliko se dr\v{z}i (ina\v{c}e nastaje \textlat{deadlock}!) i neophodno je na neki na\v{c}in zapamtiti \texttt{id} niti (ili pokaziva\v{c} na privatni semafor) kako bi se kasnije taj privatni semafor signalizirao, i time zablokirana nit deblokirala.\\\\
\textbf{\v{S}ta je tehnika prosledjivanja \v{s}tafetne palice (\textlat{passing the baton})?}\\
To je tehnika sinhronizacije gde se - umesto pu\v{s}tanja \textlat{mutex}-a - signalizira neka blokirana nit (npr. blokirana na privatnom semaforu). Time deblokirana nit ``dobija'' \textlat{mutex} od niti koja je dr\v{z}ala \textlat{mutex} i deblokirana nit mo\v{z}e koristiti podatke koje su za\v{s}ti\'{c}ene pod \textlat{mutex}-om. Nit potom mo\v{z}e da deblokira neku drugu nit i na isti na\v{c}in joj dostavi \textlat{mutex}, a mo\v{z}e i jednostavno da oslobodi \textlat{mutex}. Ukoliko ne postoji ni jedna nit kojoj \textlat{mutex} mo\v{z}e da se prosledi, neophodno je osloboditi \textlat{mutex}. \textlat{Mutex} predstavlja \v{s}tafetnu palicu koja se prosle\dj{}uje izme\dj{}u niti.
\\\\
\textbf{\large 1. na\v{c}in - sa prosle\dj{}ivanjem \v{s}tafetne palice}\\
Ovo re\v{s}enje li\v{c}i na implementaciju \texttt{Signal-and-Wait} monitora.
\begin{lstlisting}
#include <thread.h>
#include <sem.h>
#include <queue>
using namespace std;
#define N 5

typedef pair<int, int> PAIR;

Sem mutex = 1; // mutex za zastitu isWorking i qDelay
Sem privs[N] = { 0, 0, 0, 0, 0 }; // privatni semafori svih 5 niti
priority_queue<PAIR, vector<PAIR>, greater<PAIR>> qDelay;
bool isWorking = false;

void sjn_wait(int id, int prio)
{
    wait(mutex);          // uzimamo mutex da bi zastitili isWorking i qDelay
    if (isWorking) {      // ukoliko je neko u krit. sekciji blokiracemo se
        PAIR p(prio, id); // snimamo ID i prioritet.
        qDelay.push(p);   // stavljamo se u red
        signal(mutex);    // pustamo mutex
        wait(privs[id]);  // blokiramo se na privatnom semaforu
        // mutex cemo kasnije dobiti nazad od sjn_signal()
    } 
    isWorking = true;  // postavljamo da smo u kriticnoj sekciji.
    signal(mutex);     // pustamo mutex (bilo da smo ga uzeli sami ili ga dobili nazad)
}

void sjn_signal() 
{
    wait(mutex);            // uzimamo mutex
    isWorking = false;      // postavljamo da vise nismo u krit. sekciji.  
    if (! qDelay.empty()) { // ako ima nekog zablokiranog, deblokiramo ga
        PAIR p = qDelay.top();
        int dPrio = p.first;
        int dId = p.second;
        qDelay.pop();
        // signaliziramo samo privatni semafor a *ne* i mutex
        // jer hocemo da mutex dobije onaj koga signaliziramo
        signal(privs[id]);
    } else {
        // posto u ovom ovom nema nikog koga treba da budimo
        // nemamo kome da mutex prosledimo, pa ga samo pustamo.
        signal(mutex); 
    }
}

void a(int id) 
{
    int prio = rand() % 1000; // nasumicni prioritet
    while (true) {
        sjn_wait(id, prio);
        work();
        sjn_signal();
    }
}


int main() {
    Thread ta[N];
    for (int i = 0; i < N; i++)
        ta[i] = createThread(a, i);
    for (int i = 0; i < N; i++)
		join(ta[i]);
    return 0;
}

\end{lstlisting}
\textbf{\large 2. na\v{c}in - bez prosle\dj{}ivanja \v{s}tafetne palice} \\
Ovo re\v{s}enje li\v{c}i na implementaciju \texttt{Signal-and-Continue} monitora.
\begin{lstlisting}
#include <thread.h>
#include <sem.h>
#include <queue>
using namespace std;
#define N 5

typedef pair<int, int> PAIR;

Sem mutex = 1;
Sem privs[N] = { 0, 0, 0, 0, 0 };
priority_queue<PAIR, vector<PAIR>, greater<PAIR>> qDelay;
bool isWorking = false;

void sjn_wait(int id, int prio)
{
    wait(mutex);          // uzimamo mutex da bi zastitili isWorking i qDelay
    if (isWorking) {      // ukoliko je neko u krit. sekciji, blokiracemo se
        PAIR p(prio, id); // snimamo ID i prioritet.
        qDelay.push(p);   // stavljamo se u red
        signal(mutex);    // pustamo mutex
        wait(privs[id]);  // blokiramo se na privatnom semaforu
        wait(mutex)       // mutex moramo sami da uzmemo nazad 
    } 
    isWorking = true;     // postavljamo da smo u krit. sekciji
    signal(mutex);        // pustamo mutex
}


void sjn_signal() 
{
    wait(mutex);               // uzimamo mutex
    isWorking = false;         // postavljamo da vise nismo u krit. sekciji
    if (! qDelay.empty()) {    // ako ima nekog zablokiranog
        PAIR p = qDelay.top();
        int dPrio = p.first;
        int dId = p.second;
        qDelay.pop();
        signal(privs[id]);     // deblokiramo zablokiranog
    }
    // pustamo mutex bilo da smo blokirali nekog ili ne
    // ako smo deblokirali nekog, on ce sam uzeti mutex
    signal(mutex);
}

void a(int id) 
{
    int prio = rand() % 1000; // nasumicni prioritet
    while (true) {
        sjn_wait(id, prio);
        work();
        sjn_signal();
    }
}


int main() {
    Thread ta[N];
    for (int i = 0; i < N; i++)
        ta[i] = createThread(a, i);
    for (int i = 0; i < N; i++)
		join(ta[i]);
    return 0;
}

\end{lstlisting}