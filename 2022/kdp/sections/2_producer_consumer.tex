\clearpage
\section{\latin Producer and consumer}
\textbf{\large Zadatak}. Proces ima jedan deljeni ograni\v{c}en kru\v{z}ni bafer veli\v{c}ine $B$ i pokrenuo je dve niti: jednu nit \texttt{producer()} koja kreira podatke i sme\v{s}ta ih u bafer, i jednu nit \texttt{consumer()} koja dohvata podatke iz bafera i koristi ih. Ukoliko \texttt{producer} radi br\v{z}e od \texttt{consumer}-a, nastupi\'{c}e prekora\v{c}enje bafera (\textlat{buffer overflow}), dok ukoliko \texttt{consumer} radi br\v{z}e od \texttt{producer}-a, \texttt{consumer} \'{c}e po\v{c}eti da dohvata nevalidne ili nepostoja\'{c}e podatke. Koriste\'{c}i semafore, potrebno je obezbediti da se ove ne\v{z}eljene situacije ne de\v{s}avaju.
\\\\
\textbf{Napomena za re\v{s}enje:}\\
\textlat{Mutex} uvek treba da stoji naju\v{z}e mogu\'{c}e uz podatke koje \v{s}titi, a sinhronizacija uvek treba da bude oko \textlat{mutex}-a. Dakle slede\'{c}i kod za \texttt{consumer}-a je gre\v{s}ka i uzrokuje \textlat{deadlock}:
\begin{lstlisting}[language=java]
void consumer() {
    wait(mutex); 
    wait(empty);
    ...
    wait(full);
    signal(mutex);
}
\end{lstlisting}
\textbf{\large Re\v{s}enje:}
\begin{lstlisting}
#include <thread.h>
#include <sem.h>
#define B 10
int buffer[B];
int head = 0;
int tail = 0;
int count = 0;
Sem mutex = 1;
Sem full = B;
Sem empty = 0;

void producer() {
    while (true) {
        // Produce data...
        int data = rand() % 1000;
        // Put data into buffer
        wait(full);
        wait(mutex);
        buffer[head] = data;
        head = (head + 1) % B;
        count += 1;
        signal(mutex);
        signal(empty);
    }
}
void consumer() {
    while (true) {
        // Fetch data from buffer
        wait(empty);
        wait(mutex);
        int data  = buffer[tail];
        tail = (tail + 1) % B;
        count -= 1;
        signal(mutex);
        signal(full);
        // Consume data... 
        print("%d ", data);
    }
}

int main() {
    Thread ta = createThread(producer);
    Thread tb = createThread(consumer);
    join(ta);
    join(tb);
    return 0;
}

\end{lstlisting}