\clearpage
\section{Andersonov (ABQL) algoritam}
Kod \textbf{Andersonovog ABQL (Array-Based Queuing Lock) algoritma}, svaka nit \v{c}eka na svoj slot u nekom nizu \v{c}ija je veli\v{c}ina jednaka broju niti. Prvi slot je ozna\v{c}en kao otklju\v{c}an, tako da prva nit koja nai\dj{}e mo\v{z}e da prodje u kriti\v{c}nu sekciju, dok su ostali slotovi zaklju\v{c}ani. Nit atomi\v{c}no dohvata \texttt{slot} u svoj privatni \texttt{mySlot} i inkrementira globalni \texttt{slot}. Globalni \texttt{slot} se atomi\v{c}no dohvata i inkrementira pozivom specijalne instrukcije \texttt{fetch\_and\_add(x, v)}. Kada nit zavr\v{s}i sa ktriti\v{c}nom sekcijom, svoj slot resetuje, a nit zablokiranu na slede\'{c}em slotu propu\v{s}ta. Kao i Ticket algoritam, Andersonov algoritam garantuje "fer" (FIFO) redosled ulaska u kriti\v{c}nu sekciju.
\begin{lstlisting}
volatile bool flag[N] = { true };
volatile int slot = 0;

void thread(int id)
{
    while (true) {
        /* LOCK */
		int mySlot = fetch_and_add(&slot, 1) % N;
        while (flag[mySlot] == false)
            skip();
        /* CRITICAL SECTION */
        work();
        /* UNLOCK  */
        flag[mySlot] = false;
        flag[(mySlot + 1) % N] = true;
        /* NON-CRITICAL SECTION */
        usleep(50);
    }
}
\end{lstlisting}