\clearpage
\section{Cigarette smokers problem}
\textbf{\large Zadatak} Proces je pokrenuo jednu nit \texttt{agent()} koja simulira rad agenta, tri niti \texttt{smoker(MATCHES)}, \texttt{smoker(PAPER)} i \texttt{smoker(TOBACCO)} koje simuliraju rad pu\v{s}a\v{c}a, i ima deljenu strukturu \texttt{table} koja sadr\v{z}i informaciju koji predmeti se trenutno nalaze na stolu. Pu\v{s}a\v{c}i imaju beskona\v{c}ne zalihe jedne vrste predmeta, ali su im potrebene i druge dve vrste da bi pu\v{s}ili. Agent na sto nasumi\v{c}no stavlja neka dva  predmeta i \v{c}eka dok ih neki pu\v{s}a\v{c} ne uzme. Pu\v{s}a\v{c} koji vidi da mo\v{z}e da pu\v{s}i sa ta dva predmeta ih uzima sa stola i pu\v{s}i. Kada pu\v{s}a\v{c} zavr\v{s}i pu\v{s}enje, agentu signalizira da mo\v{z}e opet da stavlja predmete na sto. Implementirati pona\v{s}anje pu\v{s}a\v{c}a \texttt{void smoker(int item)} i agenta \texttt{agent()} pomo\'{c}u semafora.
\\\\
\textbf{\large Re\v{s}enje}
\begin{lstlisting}
#include <thread.h>
#include <sem.h>
#include <stdlib.h>
enum {
    MATCHES = 0,
    TOBACCO = 1,
    PAPER   = 2
};
struct Table {
    bool matches = false;
    bool paper   = false;
    bool tobacco = false;
} table; 
Sem matches = 0;
Sem tobacco = 0;
Sem paper   = 0;
Sem ok      = 0;

void agent() {
    while (true) {
        // produce items...
        int item = rand() % 3;
        // put items on the table and wait for smoker
        switch (item) {
            case MATCHES: {
                table.paper = true;
                table.tobacco = true;
                signal(matches);
                wait(ok);
            } break;
            case PAPER: {
                table.matches = true;
                table.tobacco = true;
                signal(paper);
                wait(ok);
            } break;
            case TOBACCO: {
                table.matches = true;
                table.paper = true;
                signal(tobacco);
                wait(ok);
            } break;
        }
    }
}
void smoker(int id) {
    while (true) {
        switch (id) {
            case MATCHES: {
                wait(matches);
                table.paper = false;
                table.tobacco = false;
                // create cigarette and smoke...
                // signal agent
                signal(ok);
            } break;
            case PAPER: {
                wait(paper);
                table.matches = false;
                table.tobacco = false;
                // create cigarette and smoke...
                // signal agent
                signal(ok);
            } break;
            case TOBACCO: {
                wait(tobacco);
                table.matches = false;
                table.paper = false;
                // create cigarette and smoke...
                // signal agent
                signal(ok);
            } break;
        }
    }
    
}
int main() {
    Thread ta = createThread(agent);
    Thread ts1 = createThread(smoker, MATCHES);
    Thread ts2 = createThread(smoker, PAPER);
    Thread ts3 = createThread(smoker, TOBACCO);
    join(ta);
    join(ts1);
    join(ts2);
    join(ts3);
    return 0;
}
\end{lstlisting}
\clearpage
\textbf{\large Dodatno re\v{s}enje I - u slu\v{c}aju da \texttt{agent()} ne radi signalizaciju}\\
Ako agent nikako ne signalizira pu\v{s}a\v{c}e, pu\v{s}a\v{c}i moraju da izv\v{s}avaju \texttt{busy wait} i povremeno proveravaju da li se na stolu nalaze predmeti koji su potrebni. S obzirom da vi\v{s}e niti mogu istovremeno da pristupaju stolu, neophodan je mutex.
\begin{lstlisting}
#include <thread.h>
#include <sem.h>
#include <stdlib.h>
#define ITEM_COUNT  3
#define THIRD_ITEM(first, second) (1 - (((first) - 1) + ((second) -1)))
#define OTHER_ITEM_1(item) ((item) == 0 ? ITEM_COUNT - 1 : (item) - 1)
#define OTHER_ITEM_2(item) (((item) + 1) % ITEM_COUNT) 
enum {
    MATCHES = 0,
    TOBACCO = 1,
    PAPER   = 2
};
Sem mutex = 1;
Sem ok = 0;
bool table[ITEM_COUNT] = { false, false, false };

void agent() {
    while (true) {
        // produce two items...
        int item = rand() % 3;
        // put items on the table and wait for smoker
        wait(mutex);
        table[OTHER_ITEM_1(item)] = true;
        table[OTHER_ITEM_2(item)] = true;
        signal(mutex);
        wait(ok);
    }
}
void smoker(int item) {
    int item1 = OTHER_ITEM_1(item);
    int item2 = OTHER_ITEM_2(item);
    while (true) {
        // wait for items (busy wait)
        while (true) {
            wait(mutex);
            if (table[item1] && table[item2]) {
                table[item1] = false;
                table[item2] = false;
                signal(mutex);
                break;
            } else {
                signal(mutex);
                usleep(50);
            }
        }
		// create cigarette and smoke...
		// signal agent
        signal(ok);
    }
}
int main() {
    Thread ta = createThread(agent);
    Thread ts1 = createThread(smoker, MATCHES);
    Thread ts2 = createThread(smoker, PAPER);
    Thread ts3 = createThread(smoker, TOBACCO);
    join(ta);
    join(ts1);
    join(ts2);
    join(ts3);
    return 0;
}

\end{lstlisting}
\clearpage
\textbf{\large Dodatno re\v{s}enje II - u slu\v{c}aju da \texttt{agent()} signalizira jedan semafor}\\
Recimo da agent kada stavi dva predmeta na sto signalizira semafor onog predmeta koji \textbf{nije} stavio na sto. Pu\v{s}a\v{c} tada treba da \v{c}eka na semafor predmeta koji mu \textbf{ne treba} jer se tada nalaze na stolu nalaze predmeti koji mu  trebaju. S obzirom da u ovom re\v{s}enju samo jedna nit mo\v{z}e pristupati stolu u nekom trenutku, mutex nije potreban ali nije gre\v{s}ka iskoristi ga.
\begin{lstlisting}
#include <thread.h>
#include <sem.h>
#include <stdlib.h>
#define ITEM_COUNT 3
#define OTHER_ITEM_1(item) (((item) + 1) % ITEM_COUNT)
#define OTHER_ITEM_2(item) ((item) == 0 ? (ITEM_COUNT - 1) : ((item) - 1))
enum {
    MATCHES = 0,
    TOBACCO = 1,
    PAPER   = 2
};
Sem sem[ITEM_COUNT] = { 0, 0, 0 };
Sem ok = 0;
// Sem mutex = 1;
bool table[ITEM_COUNT] = { false, false, false };

void agent() {
    while (true) {
        // produce two items...
        int item = rand() % 3;
        // put them on the table and wait for smoker
        //wait(mutex);
        table[OTHER_ITEM_1(item)] = true;
        table[OTHER_ITEM_2(item)] = true;
        //signal(mutex);
        signal(sem[item]);
        wait(ok);
    }
}
void smoker(int item) {
    int item1 = OTHER_ITEM_1(item);
    int item2 = OTHER_ITEM_2(item);
    while (true) {
		// wait for items
		wait(sem[item]);
        //wait(mutex);
		table[item1] = false;
		table[item2] = false;
        //signal(mutex);
        // create cigarette and smoke ...
		// signal agent
        signal(ok); 
    }
    
}
int main() {
    Thread ta = createThread(agent);
    Thread ts1 = createThread(smoker, MATCHES);
    Thread ts2 = createThread(smoker, PAPER);
    Thread ts3 = createThread(smoker, TOBACCO);
    join(ta);
    join(ts1);
    join(ts2);
    join(ts3);
    return 0;
}

\end{lstlisting}
\newpage
\textbf{\large Dodatno re\v{s}enje III - u slu\v{c}aju da \texttt{agent()} signalizira dva semafora}\\
Recimo da agent kada stavi dva predmeta na sto signalizira dva semafora i to ba\v{s} onih predmeta koje je i stavio na sto. U tom slu\v{c}aju, da bi svi signali bili uhva\'{c}eni, svaki pu\v{s}a\v{c} i dalje mora da \v{c}eka na semafor jednog predmeta i toj onog predmeta koji im \textbf{ne treba}. Dva pu\v{s}a\v{c}a koji uhvate signale od agenta koji treba da se dogovore da probude onog tre\'{c}eg jer je tre\'{c}i pu\v{s}a\v{c} taj koji mo\v{z}e da uzme predmete sa stola i pu\v{s}i. Mutex nije neophodan za \texttt{table[]} ali je neophodan za \texttt{agree[]} koji pu\v{s}a\v{c}i koriste da se dogovaraju na pomenut na\v{c}in.
\begin{lstlisting}
#include <thread.h>
#include <sem.h>
#include <stdlib.h>
#define ITEM_COUNT 3
#define OTHER_ITEM_1(item) (((item) + 1) % ITEM_COUNT)
#define OTHER_ITEM_2(item) ((item) == 0 ? (ITEM_COUNT - 1) : ((item) - 1))
enum {
    MATCHES = 0,
    TOBACCO = 1,
    PAPER   = 2
};
Sem sem[] = { 0, 0, 0 };
Sem ok = 0;
Sem mutex = 1;
bool table[] = { false, false, false };
bool agree[] = { false, false, false };

void agent() {
    while (true) {
        // produce two items...
        int item = rand() % 3;
        // put items on the table and wait for smoker
        table[OTHER_ITEM_1(item)] = true;
        table[OTHER_ITEM_2(item)] = true;
        signal(sem[OTHER_ITEM_1(item)]);
        signal(sem[OTHER_ITEM_2(item)]);
        wait(ok);
    }
}
void smoker(int id) {
    while (true) {
        //wait for itmes
        while (true) {
            wait(sem[id]);
            wait(mutex);
            agree[id] = true;
            if (agree[PAPER] && agree[MATCHES] && agree[TOBACCO]) {
                agree[PAPER] = false;
                agree[MATCHES] = false;
                agree[TOBACCO] = false;
                signal(mutex);
                break;
            } else if (agree[id] && agree[OTHER_ITEM_1(id)]) {
                signal(sem[OTHER_ITEM_2(id)]);
            } else if (agree[id] && agree[OTHER_ITEM_2(id)]) {
                signal(sem[OTHER_ITEM_1(id)]);
            }
            signal(mutex);
        }
        table[OTHER_ITEM_1(id)] = false;
        table[OTHER_ITEM_2(id)] = false;
        // create cigarette and smoke...
        // signal agent
        signal(ok);
    }
    
}
int main() {
    Thread ta = createThread(agent);
    Thread ts1 = createThread(smoker, MATCHES);
    Thread ts2 = createThread(smoker, PAPER);
    Thread ts3 = createThread(smoker, TOBACCO);
    join(ta);
    join(ts1);
    join(ts2);
    join(ts3);
    return 0;
}

\end{lstlisting}
\textbf{\large Dodatno re\v{s}enje IV - u slu\v{c}aju da \texttt{agent()} signalizira tri semafora.}\\
Recimo da agent signalizira semafore svih predmeta kada god stavi ne\v{s}to na sto. U tom slu\v{c}aju, pu\v{s}a\v{c} treba da \v{c}eka na jedan od semafora i samo proveri da li su predmeti koji su njemu potrebni na stolu.
\begin{lstlisting}
#include <thread.h>
#include <sem.h>
#include <stdlib.h>
#define ITEM_COUNT  3
#define THIRD_ITEM(first, second) (1 - (((first) - 1) + ((second) -1)))
#define OTHER_ITEM_1(item) ((item) == 0 ? ITEM_COUNT - 1 : (item) - 1)
#define OTHER_ITEM_2(item) (((item) + 1) % ITEM_COUNT) 
enum {
    MATCHES = 0,
    TOBACCO = 1,
    PAPER   = 2
};
Sem mutex = 1;
Sem ok = 0;
Sem sem[ITEM_COUNT] = { 0, 0, 0 };
bool table[ITEM_COUNT] = { false, false, false };

void agent() {
    while (true) {
        // produce two items
        int item = rand() % 3;
        // put them on table and wait for smoker
        wait(mutex);
        table[OTHER_ITEM_1(item)] = true;
        table[OTHER_ITEM_2(item)] = true;
        signal(mutex);
        signal(sem[MATCHES]);
        signal(sem[TOBACCO]);
        signal(sem[PAPER]);
        wait(ok);
    }
}
void smoker(int item) {
    int item1 = OTHER_ITEM_1(item);
    int item2 = OTHER_ITEM_2(item);
    while (true) {
        // wait for items
        bool spin = true;
        while (spin) {
            wait(sem[item]);
            wait(mutex);
            if (table[item1] && table[item2]) {
                table[item1] = false;
                table[item2] = false;
                spin = false;
            }
            signal(mutex);
        }
        // create cigarette and smoke...
        signal(ok);
    }
    
}
int main() {
    Thread ta = createThread(agent);
    Thread ts1 = createThread(smoker, MATCHES);
    Thread ts2 = createThread(smoker, PAPER);
    Thread ts3 = createThread(smoker, TOBACCO);
    join(ta);
    join(ts1);
    join(ts2);
    join(ts3);
    return 0;
}

\end{lstlisting}