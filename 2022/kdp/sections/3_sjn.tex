\clearpage
\section{\latin Shortest job next}
\textbf{\large Zadatak} Proces je pokrenuo $N$ niti koje treba pristupe kriti\v{c}noj sekciji po prioritetnom redosledu. Naime, ukoliko postoje vi\v{s}e niti koje istovremeno \v{z}ele da udju u kriti\v{c}nu sekciju, prva ulazi ona koja ima najmanje posla. Implementirati ovakvo pona\v{s}anje niti pomo\'{c}u uslovnih kriti\v{c}nih regiona.
\\\\
\textbf{\large Re\v{s}enje}\\
Ukoliko je kriti\v{c}na sekcija zauzeta, mo\v{z}emo staviti sebe u prioritetni red i sa\v{c}ekati dok slede\'{c}i u redu ne bude ba\v{s} na\v{s} \texttt{id}.
\begin{lstlisting}
#include <thread.h>
#include <region.h>
#include <stdlib.h>
#include <queue>
using namespace std;
#define N 10

typedef pair<int, int> PAIR;

struct SJN : Region {
    bool locked = false;
    priority_queue<PAIR, vector<PAIR>, greater<PAIR>> qEnter;
} sjn;

void a(int id) 
{
    int prio = rand() % 1000; // nasumicni prioritet
    while (true) {
        region (sjn) {
            sjn.qEnter.push(PAIR(prio, id));
            await(sjn.locked == false && sjn.qEnter.top().second == id);
            sjn.qEnter.pop();
            sjn.locked = true;
        }
        work();
        region (sjn) {
            sjn.locked = false;
        }
    }
}


int main() 
{
    Thread tp[N];
    for (int i = 0; i < N; i++) 
        tp[i] = createThread(a, i);
    for (int i = 0; i < N; i++) 
        join(tp[i]);
    return 0;
}

\end{lstlisting}