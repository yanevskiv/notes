\clearpage
\section{\latin Sleeping barber}
\textbf{\large Zadatak} Proces je pokrenuo jednu nit koja simulira pona\v{s}anje uspavanog berberina i $N$ niti koje simuliraju pona\v{s}anje mu\v{s}terija koje dolaze kod berberina da se \v{s}i\v{s}aju. Berberin spava dok mu ne dodje mu\v{s}terija u berbernicu. Mu\v{s}terija kada dodje u berbernicu treba da probudi berberina i sedne u red \v{c}ekanja dok je berberin ne pozove na \v{s}i\v{s}anje. Kada berberin zavr\v{s}i sa \v{s}i\v{s}anjem mu\v{s}terije, mu\v{s}terja pla\'{c}a berberinu i odlazi. Ukoliko u berbernici nema mesta za sedenje, mu\v{s}terija odlazi i vra\'{c}a se nakon nekog vremena. Pomo\'{c}u semafora implementirati metode \texttt{void barber()} i \texttt{void customer(int id)} koje respektivno opona\v{s}aju berberina i mu\v{s}terije.
\\\\
\textbf{\large Re\v{s}enje}
\begin{lstlisting}
#include <thread.h>
#include <sem.h>
#include <queue>
#define MAX_SEATS 5
#define N 10
Sem privb = 0; // privatni semafor berberina
Sem privc[N];  // privatni semafori musterija
Sem mutex = 1; // mutex koji stiti qSeats
std::queue<int> qSeats;
extern void cut_hair(int custId);

void barber() {
    while (true) {
        wait(privb);
        wait(mutex);
        int custId = qSeats.front();
        qSeats.pop();
        signal(mutex);
        signal(privc[custId]);
        cut_hair(custId);
    }
}
void customer(int id) {
    while (true) {
        wait(mutex);
        if (qSeats.size() < MAX_SEATS)  {
            qSeats.push(id);
            signal(mutex);
            wait(privc[id]);
            cut_hair(id);
        } else {
            signal(mutex);
        }
        usleep(100);
    }
}
int main() {
    for (int i = 0; i < N; i++) 
        init(privc[i], 0);
    Thread tb = createThread(barber);
    Thread tc[N];
    for (int i = 0; i < N; i++) 
        tc[i] = createThread(customer, i);
    join(tb);
    join(tb);
    return 0;
}

\end{lstlisting}