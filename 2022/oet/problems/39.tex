\textbf{\Large 39.} Malo nepokretno telo nalazi se u vakuumu, a njegovo naelektrisanje je $Q = 1\ \mu\mathrm C$. Koliki rad izvr\v{s}i elektri\v{c}na sila pri preme\v{s}tanju drugog malog tela (probnog naelektrisanja), mase $m = 1\ \mu g$ i naelektrisanja $Q_p = 1\ \mathrm{pC}$, (a) iz veoma udaljene ta\v{c}ke u ta\v{c}ku koja je na odstojanju $r_1 = 100\ \mathrm{mm}$ od nepokretnog tela i (b) iz ta\v{c}ke koja je na odstojanju $r_1$ od nepokretnog tela u ta\v{c}ku koja je na odstojanju $r_2 = 1\ \mathrm m$?
\\\\
\textbf{\Large Re\v{s}enje}\\
\textbf{\Large 39.} Malo nepokretno telo nalazi se u vakuumu, a njegovo naelektrisanje je $Q = 1\ \mu\mathrm C$. Koliki rad izvr\v{s}i elektri\v{c}na sila pri preme\v{s}tanju drugog malog tela (probnog naelektrisanja), mase $m = 1\ \mu g$ i naelektrisanja $Q_p = 1\ \mathrm{pC}$, (a) iz veoma udaljene ta\v{c}ke u ta\v{c}ku koja je na odstojanju $r_1 = 100\ \mathrm{mm}$ od nepokretnog tela i (b) iz ta\v{c}ke koja je na odstojanju $r_1$ od nepokretnog tela u ta\v{c}ku koja je na odstojanju $r_2 = 1\ \mathrm m$?
\\\\
\textbf{\Large Re\v{s}enje}\\
\textbf{\Large 39.} Malo nepokretno telo nalazi se u vakuumu, a njegovo naelektrisanje je $Q = 1\ \mu\mathrm C$. Koliki rad izvr\v{s}i elektri\v{c}na sila pri preme\v{s}tanju drugog malog tela (probnog naelektrisanja), mase $m = 1\ \mu g$ i naelektrisanja $Q_p = 1\ \mathrm{pC}$, (a) iz veoma udaljene ta\v{c}ke u ta\v{c}ku koja je na odstojanju $r_1 = 100\ \mathrm{mm}$ od nepokretnog tela i (b) iz ta\v{c}ke koja je na odstojanju $r_1$ od nepokretnog tela u ta\v{c}ku koja je na odstojanju $r_2 = 1\ \mathrm m$?
\\\\
\textbf{\Large Re\v{s}enje}\\
\textbf{\Large 39.} Malo nepokretno telo nalazi se u vakuumu, a njegovo naelektrisanje je $Q = 1\ \mu\mathrm C$. Koliki rad izvr\v{s}i elektri\v{c}na sila pri preme\v{s}tanju drugog malog tela (probnog naelektrisanja), mase $m = 1\ \mu g$ i naelektrisanja $Q_p = 1\ \mathrm{pC}$, (a) iz veoma udaljene ta\v{c}ke u ta\v{c}ku koja je na odstojanju $r_1 = 100\ \mathrm{mm}$ od nepokretnog tela i (b) iz ta\v{c}ke koja je na odstojanju $r_1$ od nepokretnog tela u ta\v{c}ku koja je na odstojanju $r_2 = 1\ \mathrm m$?
\\\\
\textbf{\Large Re\v{s}enje}\\
\input{solutions/39.tex}