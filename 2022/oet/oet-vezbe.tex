\documentclass{report}
\usepackage[a4paper, margin=2.3cm]{geometry}
\usepackage[utf8]{inputenc}
\usepackage{titlesec}
\usepackage{cmsrb}
\usepackage{subfigure}
\usepackage{tikz}
\usepackage{xcolor}
\usepackage{caption}
\usepackage{amsmath}
\usepackage{amsfonts}
\usepackage{amssymb}
\usepackage{pgfplots}
\usepackage{cancel}
\usepackage{url}
\usepackage{wrapfig}
\usepackage[hidelinks]{hyperref}
\pgfplotsset{compat=1.18}
\usetikzlibrary{arrows.meta}
\title{Osnovi elektrotehnike 1 - Zadaci sa ve\v{z}bi - Skripta}
\author{Marko \v{Z}uti\'{c}, Ivan Janevski}
\date{Oktobar 2022}
\renewcommand{\contentsname}{Sadr\v{z}aj}
\titleformat{\section}[display]{\normalfont\bfseries}{}{0pt}{\huge}
\captionsetup{labelformat=empty}

% Cyrillic settings
\usepackage[T1,OT2]{fontenc}
\usepackage[serbian]{babel}
\newcommand\textcyr[1]{{\fontencoding{OT2}\fontfamily{wncyr}\selectfont #1}}
\newcommand\textlat[1]{{\fontencoding{T1}\fontfamily{wncyr}\selectfont #1}}
\newcommand\cyrillic{\fontencoding{OT2}\fontfamily{wncyr}\selectfont}
\newcommand\latin{\fontencoding{T1}\fontfamily{wncyr}\selectfont}
\newenvironment{nocyrillic}{\fontencoding{T1}\fontfamily{wncyr}\selectfont}{\fontencoding{OT2}\fontfamily{wncyr}\selectfont}
\begin{document}
\maketitle
\newpage
\hspace{0pt}
\vfill
\begin{center}
Ova skripta je otvorenog koda (\textlat{GNU Free Documentation License}) \v{c}iji se izvori\v{s}ni kod mo\v{z}e na\'{c}i na linku:\\ \textlat{\url{https://github.com/yanevskiv/notes/tree/master/2022/oet}}
\end{center}
\vfill
\hspace{0pt}
\newpage
\tableofcontents{}
\newpage
\titleformat{\chapter}[display]{\normalfont\bfseries}{}{0pt}{\Huge}
\chapter{Ve\v{z}be}

\begin{enumerate}
    \item tro\v{c}as
    \begin{itemize}
        \item Kulonov zakon: 4, 6, (5)
        \item Gustine naelektrisanja: 14, 15, 17, 18, (16)
        \item Vektor ja\v{c}ine elektri\v{c}nog polja: 9, 22 (19, 21)
    \end{itemize}
    \item tro\v{c}as
    \begin{itemize}
        \item 23 $+$ 25, 28, 32, 34, 37 (24, 29, 30, 36)
        \item Potencijal: 39, 48, 51
    \end{itemize}
\end{enumerate}

\clearpage
\section{1. nedelja}
\textbf{\Large 4.} Tri mala tela, naelektrisanja $Q_1=Q_2 = 20\ \mathrm{pC}$ i $Q_3 = -50\ \mathrm{pC}$, nalaze se u vazduhu u ta\v{c}kama $A(-a, 0, 0),\ b(a, 0, 0)$ i $M(0,2a, 0)$, gde je  $a = 0,2\ \mathrm m$, kao na slici 4.1. Izra\v{c}unati vektor sile na telo naelektrisanja $Q_3$.
\begin{figure}[h]
    \centering
    \begin{tikzpicture}
    % Coordinates
    \coordinate (O) at (0, 0);
    \coordinate (A) at (-2, 0);
    \coordinate (B) at (2, 0);
    \coordinate (M) at (0, 4);
    
    % Coordinate system
    \draw[-Latex] (-3, 0) -- (3, 0) node [below] {$x$};
    \draw[-Latex] (0, -.5) -- (0, 4.5) node [right] {$y$};
    
    % Epsilon 0
    \node at (-3, 4) {$\varepsilon_0$};
    
    % Point charges and origin
    \draw (A) node[circle, fill, inner sep =1.5] {} 
        node [above, inner sep = 10] {$A$} 
        node [below right, inner sep = 5] {$Q_1$} 
        node [below left, inner sep = 5] {$-a$}; 
    \draw (B) node[circle, fill, inner sep = 1.5] {} 
        node [above, inner sep = 10] {$B$} 
        node [below right, inner sep = 5] {$Q_2$} 
        node [below left, inner sep = 5] {$a$}; 
    \draw (M) node[circle, fill, inner sep = 1.5] {} 
        node [right, inner sep = 5] {$M$} 
        node [above left, inner sep = 4] {$Q_3$} 
        node [below left, inner sep = 4] {$2a$}; 
    \draw (O) node[circle, draw, fill=white, inner sep = 3] {}
        node [circle, fill=black, inner sep = 1.5 ]{} 
        node[above left] {$O$}
        node[below right] {$z$};
\end{tikzpicture}

    \caption{Slika 4.1.}
\end{figure}
\\
\textbf{\Large Re\v{s}enje}\\
\textbf{\Large 4.} Tri mala tela, naelektrisanja $Q_1=Q_2 = 20\ \mathrm{pC}$ i $Q_3 = -50\ \mathrm{pC}$, nalaze se u vazduhu u ta\v{c}kama $A(-a, 0, 0),\ b(a, 0, 0)$ i $M(0,2a, 0)$, gde je  $a = 0,2\ \mathrm m$, kao na slici 4.1. Izra\v{c}unati vektor sile na telo naelektrisanja $Q_3$.
\begin{figure}[h]
    \centering
    \begin{tikzpicture}
    % Coordinates
    \coordinate (O) at (0, 0);
    \coordinate (A) at (-2, 0);
    \coordinate (B) at (2, 0);
    \coordinate (M) at (0, 4);
    
    % Coordinate system
    \draw[-Latex] (-3, 0) -- (3, 0) node [below] {$x$};
    \draw[-Latex] (0, -.5) -- (0, 4.5) node [right] {$y$};
    
    % Epsilon 0
    \node at (-3, 4) {$\varepsilon_0$};
    
    % Point charges and origin
    \draw (A) node[circle, fill, inner sep =1.5] {} 
        node [above, inner sep = 10] {$A$} 
        node [below right, inner sep = 5] {$Q_1$} 
        node [below left, inner sep = 5] {$-a$}; 
    \draw (B) node[circle, fill, inner sep = 1.5] {} 
        node [above, inner sep = 10] {$B$} 
        node [below right, inner sep = 5] {$Q_2$} 
        node [below left, inner sep = 5] {$a$}; 
    \draw (M) node[circle, fill, inner sep = 1.5] {} 
        node [right, inner sep = 5] {$M$} 
        node [above left, inner sep = 4] {$Q_3$} 
        node [below left, inner sep = 4] {$2a$}; 
    \draw (O) node[circle, draw, fill=white, inner sep = 3] {}
        node [circle, fill=black, inner sep = 1.5 ]{} 
        node[above left] {$O$}
        node[below right] {$z$};
\end{tikzpicture}

    \caption{Slika 4.1.}
\end{figure}
\\
\textbf{\Large Re\v{s}enje}\\
\textbf{\Large 4.} Tri mala tela, naelektrisanja $Q_1=Q_2 = 20\ \mathrm{pC}$ i $Q_3 = -50\ \mathrm{pC}$, nalaze se u vazduhu u ta\v{c}kama $A(-a, 0, 0),\ b(a, 0, 0)$ i $M(0,2a, 0)$, gde je  $a = 0,2\ \mathrm m$, kao na slici 4.1. Izra\v{c}unati vektor sile na telo naelektrisanja $Q_3$.
\begin{figure}[h]
    \centering
    \input{tikz/4_threedots.tex}
    \caption{Slika 4.1.}
\end{figure}
\\
\textbf{\Large Re\v{s}enje}\\
\input{solutions/4.tex}
\begin{align*}
    |\mathbf{F_{12}}| = |\mathbf{F_{21}}| = \frac{1}{4\pi\varepsilon_0} |Q_1Q_2| 
    = \frac{Q_1(Q-Q_1)}{4\pi\varepsilon_0 d^2}
\end{align*}
\begin{align*}
    |\mathbf{F}| = \frac{1}{4\pi\varepsilon_0d^2}Q\cdot \frac{Q_1}{Q_2}\left(Q - Q\cdot\frac{Q_1}{Q_2}\right) 
    = \frac{Q^2}{4\pi\varepsilon_0d^2}\frac{Q_1}{Q}\left(1 - \frac{Q_1}{Q}\right)
\end{align*}
\begin{align*}
    f\left(\frac{Q_1}{Q}\right) = \frac{Q_1}{Q}\left(1 - \frac{Q_1}{Q}\right),\; 0 \leq \frac{Q_1}{Q} \leq 1
\end{align*}
\begin{align*}
    \boxed{\left(\frac{Q_1}{Q}\right)_\mathrm{opt} = 0,5},\ 
    \boxed{\mathbf{|F|}_\mathrm{max} = \frac{Q^2}{16\pi\varepsilon_0 d^2}}
\end{align*}
\begin{figure}[h]
    \centering
    \begin{figure}[h]
    \centering
    \begin{tikzpicture}
        \begin{axis}[
            axis y line=center,
            axis x line=middle,
            axis equal,
            %grid=both,
            xmax=1.5,xmin=-0.5,
            ymin=-0.5,ymax=0.5,
            xlabel=$x$,ylabel=$f$,
            %xtick={-10,...,10},
            %ytick={-10,...,10},
            width=15cm,
        ]
        % plot function
        \addplot[samples=150, mark=none, blue] {x * (1 - x)};
        % x mark
        \draw[dashed] (0.5, 0) node [below, outer sep = 5] {$0.5$} -- (0.5, 0.25);
        % y mark
        \draw[dashed] (0, 0.25) node [left, outer sep = 5] {$0.25$} -- (0.5, 0.25);
        \end{axis}
    \end{tikzpicture}
\end{figure}
\end{figure}
\textbf{\Large 9.} Polo\v{z}aji tri mala tela naelektrisanja $Q$, $Q$ i $-2Q$ odre\dj{}eni su ta\v{c}kama $A(a,0,0)$, $B(0, a, 0)$ i $C(0, 0, a)$, $a > 0$, respektirvno (slika 9.1). Sredina je vazduh. Odrediti izraze za: (a) vektor ja\v{c}ine elektri\v{c}nog polja u ta\v{c}ki $M(a, a, a)$ i (b) intenzitet elektri\v{c}nog polja u istoj ta\v{c}ki.
\\\\
\textbf{\Large Re\v{s}enje}
\textbf{\Large 9.} Polo\v{z}aji tri mala tela naelektrisanja $Q$, $Q$ i $-2Q$ odre\dj{}eni su ta\v{c}kama $A(a,0,0)$, $B(0, a, 0)$ i $C(0, 0, a)$, $a > 0$, respektirvno (slika 9.1). Sredina je vazduh. Odrediti izraze za: (a) vektor ja\v{c}ine elektri\v{c}nog polja u ta\v{c}ki $M(a, a, a)$ i (b) intenzitet elektri\v{c}nog polja u istoj ta\v{c}ki.
\\\\
\textbf{\Large Re\v{s}enje}
\textbf{\Large 9.} Polo\v{z}aji tri mala tela naelektrisanja $Q$, $Q$ i $-2Q$ odre\dj{}eni su ta\v{c}kama $A(a,0,0)$, $B(0, a, 0)$ i $C(0, 0, a)$, $a > 0$, respektirvno (slika 9.1). Sredina je vazduh. Odrediti izraze za: (a) vektor ja\v{c}ine elektri\v{c}nog polja u ta\v{c}ki $M(a, a, a)$ i (b) intenzitet elektri\v{c}nog polja u istoj ta\v{c}ki.
\\\\
\textbf{\Large Re\v{s}enje}
\input{solutions/9.tex}
\begin{figure}[h]
    \centering
    \begin{tikzpicture}
    \draw[gray,-Latex] (-3, 0) -- (3, 0) node[below] {$x$};
    \draw[line width=2, black] (-2, 0) 
        node[below]{-$\frac{a}{2}$}  node[above right]{$Q'(x)$} 
    --  node[below] {$0$} (2, 0) 
        node[below]{$\frac{a}{2}$};
    \draw[line width=1, black] (-2,-0.1) -- (-2, 0.1);
    \draw[line width=1, black] (0,-0.1) -- (0, 0.1);
    \draw[line width=1, black] (2,-0.1) -- (2, 0.1);
    \draw[line width=1, blue] (1.5,-0.25) 
    -- node[above, outer sep = 10]{$\mathrm d x$} 
       node[below , outer sep = 10]{$\mathrm d Q$} 
       (1.5, 0.25);
    \draw[blue, -Latex] (1.8, 0.15) -- (1.5, 0.15);
    \draw[blue, -Latex] (1.1, 0.15) -- (1.4, 0.15);
    \draw[line width=1, blue] (1.4,-0.25) -- (1.4, 0.25);
    \draw[line width=2, blue] (1.5, 0) -- (1.4, 0);
\end{tikzpicture}

\end{figure}
\\
Odlu\v{c}imo se da integralimo u smeru $x$-ose jer je tada $\mathrm d\ell = \mathrm d x$
\begin{align*}
    Q' = \frac{\mathrm d Q}{\mathrm d\ell},\; \mathrm d \ell > 0\\
    Q' = \frac{\mathrm d Q}{\mathrm d x},\; \underline{\mathrm d x > 0}
\end{align*}
Postavimo linijski integral:
\begin{align*}
    \mathrm dQ = Q'\mathrm dx 
    && Q = \int_L Q'\mathrm dx = \int_{-a/2}^{a/2} 8Q_0'\frac{|x^3|}{a^3}\mathrm dx
    && {\color{red} |x^3| = \begin{cases}
        -x^3, & x < 0\\
        x^3, & x > 0
    \end{cases}}
\end{align*}
Re\v{s}imo integral u sredini:
\begin{align*}
Q = \frac{8Q_0'}{a^3}\int_{-a/2}^{a/2} |x^3|\mathrm d x 
= \frac{8 Q_0'}{a^3} \underbrace{2\cdot\int_0^{a/2} x^3\mathrm d x}_{\color{red}\int_{-a/2}^0  (-x^3)\mathrm dx + \int_0^{a/2} x^3\mathrm d x} = \frac{\cancelto{4}{16}Q_0'}{a^3}\frac{x^4}{\cancel{4}}\big|_0^{a/2} = \frac{4Q_0'}{\cancel{a^3}}\frac{\cancel{a^4}}{16} = \frac{Q_0' a}{4}
\end{align*}
Kona\v{c}no re\v{s}enje: $\boxed{Q = \frac{Q_0' a}{4}}$

\begin{figure}[h]
    \centering
    \begin{tikzpicture}
    % shade rectangle
    \draw[fill=orange!40] (0,0) rectangle (2,1);

    % draw dx
    \draw[black,fill=magenta!60] (1,0) rectangle (1.1, 1);
    \node[magenta] at (1,0) [below left, outer sep=2] {\tiny $x$};
    \node[magenta] at (1,0) [below right] {\tiny $x + \mathrm dx$};
    \draw[magenta, -Latex] (0.7, 0.2) -- (1, 0.2);
    \draw[magenta, -Latex] (1.4, 0.2) node [above ]{\tiny $\mathrm dx$} -- (1.1, 0.2);
    \node[magenta] at (1,1) [above] {\tiny $\mathrm dQ, \mathrm dS$};
    
    % y-axis
    \draw[-Latex] (0, -0.5) -- (0, 1.5) node [above left] {$y$};
    
    % x-axis
    \draw [-Latex] (-0.5, 0) -- (2.5, 0) node [below right] {$x$};
    
    % origin
    \node at (0, 0) [below left] {$O$};
    
    % b-line
    \draw (-0.1,1) node [left]{$b$} -- (2,1);
    
    % a-line
    \draw (2, -0.1) node[below] {$a$} -- (2, 1);
    
    % rho_s
    \node at (0.3, 0.75) {$\mathrm{\rho}_S$};
    
\end{tikzpicture}

\end{figure}
\begin{align*}
    \rho_S &= \frac{\mathrm dQ}{\mathrm dS}, \mathrm dS > 0\\
    \mathrm dQ &= \rho_S\mathrm dS\\
    Q &= \int_S\rho_S\mathrm dS
\end{align*}
\begin{align*}
    Q = \int_0^a kxb\cdot \mathrm dx = kb\int_0^a x\mathrm dx = kb\frac{a^2}{2}, k = \frac{2Q}{ba^2}\\
    \boxed{\rho_S = k\cdot x = \frac{2Qx}{a^2 b}, \begin{array}{c}
         0 \leq x \leq a \\
         0 \leq y \leq b 
    \end{array}}    
\end{align*}
\textbf{\Large 17.} U lopti polupre\v{c}nika $a$ neravnomerno su raspodeljena naelektrisanja tako da je njihova zapreminska gustina data izrazom $\rho(r) = \rho_0 = \frac{a - r}{a}, r\in[0, a]$, gde je $r$ odstojanje posmatrane ta\v{c}ke od centra lopte, a $\rho_0$ konstantna veli\v{c}ina. koliko je ukupno naelektrisanje lopte?
\\\\
\textbf{\Large Re\v{s}enje}
\textbf{\Large 17.} U lopti polupre\v{c}nika $a$ neravnomerno su raspodeljena naelektrisanja tako da je njihova zapreminska gustina data izrazom $\rho(r) = \rho_0 = \frac{a - r}{a}, r\in[0, a]$, gde je $r$ odstojanje posmatrane ta\v{c}ke od centra lopte, a $\rho_0$ konstantna veli\v{c}ina. koliko je ukupno naelektrisanje lopte?
\\\\
\textbf{\Large Re\v{s}enje}
\textbf{\Large 17.} U lopti polupre\v{c}nika $a$ neravnomerno su raspodeljena naelektrisanja tako da je njihova zapreminska gustina data izrazom $\rho(r) = \rho_0 = \frac{a - r}{a}, r\in[0, a]$, gde je $r$ odstojanje posmatrane ta\v{c}ke od centra lopte, a $\rho_0$ konstantna veli\v{c}ina. koliko je ukupno naelektrisanje lopte?
\\\\
\textbf{\Large Re\v{s}enje}
\input{solutions/17.tex}
\begin{align*}
    \rho = \frac{\mathrm Q}{\mathrm dV},
    Q' = \frac{Q_L}{L}\\
\end{align*}
\begin{align*}
    \mathrm dQ_L &= \rho\mathrm dV = \rho_0\frac{y}{b} aL\mathrm dy\\
    Q_L &= \int_0^b\rho_0\frac{aL}{b}y\mathrm dy = \frac{\rho_0 aL}{b}\frac{b^2}{2}\\
    Q_L &= \frac{\rho_0 a b L}{2},\,
    Q' = \frac{Q_L}{L} = \frac{\rho_0 ab}{2}
\end{align*}

\textbf{\Large 22.} Temena trougla prikazanog na slici 22.1 su u ta\v{c}kama $A(0, 0, 0)$, $B(a, 0, 0)$ i $C(0, a, 0)$. Stranice trougla su ravnomerno naelektrisane naelektrisanjem podu\v{z}ne gustine $Q'$. Sredina je vazduh. Odrediti vektor elektri\v{c}nog polja u ta\v{c}ki $M(a, a, 0)$
\\\\
\textbf{\Large Re\v{s}enje}
\textbf{\Large 22.} Temena trougla prikazanog na slici 22.1 su u ta\v{c}kama $A(0, 0, 0)$, $B(a, 0, 0)$ i $C(0, a, 0)$. Stranice trougla su ravnomerno naelektrisane naelektrisanjem podu\v{z}ne gustine $Q'$. Sredina je vazduh. Odrediti vektor elektri\v{c}nog polja u ta\v{c}ki $M(a, a, 0)$
\\\\
\textbf{\Large Re\v{s}enje}
\textbf{\Large 22.} Temena trougla prikazanog na slici 22.1 su u ta\v{c}kama $A(0, 0, 0)$, $B(a, 0, 0)$ i $C(0, a, 0)$. Stranice trougla su ravnomerno naelektrisane naelektrisanjem podu\v{z}ne gustine $Q'$. Sredina je vazduh. Odrediti vektor elektri\v{c}nog polja u ta\v{c}ki $M(a, a, 0)$
\\\\
\textbf{\Large Re\v{s}enje}
\input{solutions/22.tex}

\clearpage
\section{2. nedelja}
\textbf{\Large 23.} Vrlo duga\v{c}ka tanka ravna traka, \v{s}irine $a$, ravnomerno je naelektrisana naelektrisanjem povr\v{s}inske gustine $\rho_{\mathrm S}$, a nalazi se u vazduhu. (a) Odrediti izraz za vektor elektri\v{c}nog polja u projzvoljnoj ta\v{c}ki u okolini ove trake. (b) Na osnovu izraza dobijenog pod (a), izvesti izraz za vektor elektri\v{c}nog polja u okolini naelektrisane ravni u vazduhu.
\\\\
\textbf{\Large Re\v{s}enje}\\
\textbf{\Large 23.} Vrlo duga\v{c}ka tanka ravna traka, \v{s}irine $a$, ravnomerno je naelektrisana naelektrisanjem povr\v{s}inske gustine $\rho_{\mathrm S}$, a nalazi se u vazduhu. (a) Odrediti izraz za vektor elektri\v{c}nog polja u projzvoljnoj ta\v{c}ki u okolini ove trake. (b) Na osnovu izraza dobijenog pod (a), izvesti izraz za vektor elektri\v{c}nog polja u okolini naelektrisane ravni u vazduhu.
\\\\
\textbf{\Large Re\v{s}enje}\\
\textbf{\Large 23.} Vrlo duga\v{c}ka tanka ravna traka, \v{s}irine $a$, ravnomerno je naelektrisana naelektrisanjem povr\v{s}inske gustine $\rho_{\mathrm S}$, a nalazi se u vazduhu. (a) Odrediti izraz za vektor elektri\v{c}nog polja u projzvoljnoj ta\v{c}ki u okolini ove trake. (b) Na osnovu izraza dobijenog pod (a), izvesti izraz za vektor elektri\v{c}nog polja u okolini naelektrisane ravni u vazduhu.
\\\\
\textbf{\Large Re\v{s}enje}\\
\input{solutions/23.tex}
\textbf{\Large 25.} Na slici 25.1 je prikazan veoma duga\v{c}ak tanak naelektrisani ugaonik (diedar) u vazduhu. Smatraju\'{c}i da je ugaonik ravnomerno naelektrisan po svojoj povr\v{s}i naelektrisanjem povr\v{s}inske gustine $\rho_S$, odrediti vektor elektri\v{c}nog polja u ta\v{c}ki $M(a,a,0), a>0$.
\\\\
\textbf{\Large Re\v{s}enje}\\
\textbf{\Large 25.} Na slici 25.1 je prikazan veoma duga\v{c}ak tanak naelektrisani ugaonik (diedar) u vazduhu. Smatraju\'{c}i da je ugaonik ravnomerno naelektrisan po svojoj povr\v{s}i naelektrisanjem povr\v{s}inske gustine $\rho_S$, odrediti vektor elektri\v{c}nog polja u ta\v{c}ki $M(a,a,0), a>0$.
\\\\
\textbf{\Large Re\v{s}enje}\\
\textbf{\Large 25.} Na slici 25.1 je prikazan veoma duga\v{c}ak tanak naelektrisani ugaonik (diedar) u vazduhu. Smatraju\'{c}i da je ugaonik ravnomerno naelektrisan po svojoj povr\v{s}i naelektrisanjem povr\v{s}inske gustine $\rho_S$, odrediti vektor elektri\v{c}nog polja u ta\v{c}ki $M(a,a,0), a>0$.
\\\\
\textbf{\Large Re\v{s}enje}\\
\input{solutions/25.tex}
\textbf{\Large 28.} Svileno vlakno oblika polukruga polupre\v{c}nika $a$ ravnomerno je naelektrisano naelektrisanjem $Q$ i nalazi se u vazduhu, kao na slici 28.1. Odrediti vektor elektri\v{c}nog polja u ta\v{c}ki $M(0,0,z)$.
\\\\
\textbf{\Large Re\v{s}enje}\\
\textbf{\Large 28.} Svileno vlakno oblika polukruga polupre\v{c}nika $a$ ravnomerno je naelektrisano naelektrisanjem $Q$ i nalazi se u vazduhu, kao na slici 28.1. Odrediti vektor elektri\v{c}nog polja u ta\v{c}ki $M(0,0,z)$.
\\\\
\textbf{\Large Re\v{s}enje}\\
\textbf{\Large 28.} Svileno vlakno oblika polukruga polupre\v{c}nika $a$ ravnomerno je naelektrisano naelektrisanjem $Q$ i nalazi se u vazduhu, kao na slici 28.1. Odrediti vektor elektri\v{c}nog polja u ta\v{c}ki $M(0,0,z)$.
\\\\
\textbf{\Large Re\v{s}enje}\\
\input{solutions/28.tex}
\textbf{\Large 32.} Polusferna povr\v{s} polupre\v{c}nika $a$ ravnomerno je naelektrisana naelektrisanjem $Q$ i nalazi se u vazduhu (slika 32.1), Odrediti vektor elektri\v{c}nog polja u koordinatnom po\v{c}etku.
\\\\
\textbf{\Large Re\v{s}enje}\\
\textbf{\Large 32.} Polusferna povr\v{s} polupre\v{c}nika $a$ ravnomerno je naelektrisana naelektrisanjem $Q$ i nalazi se u vazduhu (slika 32.1), Odrediti vektor elektri\v{c}nog polja u koordinatnom po\v{c}etku.
\\\\
\textbf{\Large Re\v{s}enje}\\
\textbf{\Large 32.} Polusferna povr\v{s} polupre\v{c}nika $a$ ravnomerno je naelektrisana naelektrisanjem $Q$ i nalazi se u vazduhu (slika 32.1), Odrediti vektor elektri\v{c}nog polja u koordinatnom po\v{c}etku.
\\\\
\textbf{\Large Re\v{s}enje}\\
\input{solutions/32.tex}
\textbf{\Large 34.} Na slici 34.1 prikazana je duga\v{c}ka traka savijena u obliku polovine veoma duga\v{c}kog kru\v{z}nog cilindra polupre\v{c}nika $a$ i veoma duga\v{c}ak pravolinijski provodnik postavljen na osi cilindra. Traka i pravolinijski provodnik ravnomerno su naelektrisani naelektrisanjem podu\v{z}ne gustine $Q'$. Sredina je vazduh. Odrediti podu\v{z}ne elektri\v{c}ne sile na pravolinijski provodnik.
\\\\
\textbf{\Large Re\v{s}enje}\\
\textbf{\Large 34.} Na slici 34.1 prikazana je duga\v{c}ka traka savijena u obliku polovine veoma duga\v{c}kog kru\v{z}nog cilindra polupre\v{c}nika $a$ i veoma duga\v{c}ak pravolinijski provodnik postavljen na osi cilindra. Traka i pravolinijski provodnik ravnomerno su naelektrisani naelektrisanjem podu\v{z}ne gustine $Q'$. Sredina je vazduh. Odrediti podu\v{z}ne elektri\v{c}ne sile na pravolinijski provodnik.
\\\\
\textbf{\Large Re\v{s}enje}\\
\textbf{\Large 34.} Na slici 34.1 prikazana je duga\v{c}ka traka savijena u obliku polovine veoma duga\v{c}kog kru\v{z}nog cilindra polupre\v{c}nika $a$ i veoma duga\v{c}ak pravolinijski provodnik postavljen na osi cilindra. Traka i pravolinijski provodnik ravnomerno su naelektrisani naelektrisanjem podu\v{z}ne gustine $Q'$. Sredina je vazduh. Odrediti podu\v{z}ne elektri\v{c}ne sile na pravolinijski provodnik.
\\\\
\textbf{\Large Re\v{s}enje}\\
\input{solutions/34.tex}
\textbf{\Large 37.} U lopti polupre\v{c}nika $a$, u vazduhu, raspodeljeno je naelektrisanje zapreminske gustine $\rho(r) = \rho(a)\frac{r}{a}, 0\leq r\leq a$, gde je $r$ odstojanje posmatrane ta\v{c}ke od sredi\v{s}ta sfere. Koriste\'{c}i se principom superpozicije, odrediti vektor elektri\v{c}nog polja ovog naelektrisanja.
\\\\
\textbf{\Large Re\v{s}enje}\\
\textbf{\Large 37.} U lopti polupre\v{c}nika $a$, u vazduhu, raspodeljeno je naelektrisanje zapreminske gustine $\rho(r) = \rho(a)\frac{r}{a}, 0\leq r\leq a$, gde je $r$ odstojanje posmatrane ta\v{c}ke od sredi\v{s}ta sfere. Koriste\'{c}i se principom superpozicije, odrediti vektor elektri\v{c}nog polja ovog naelektrisanja.
\\\\
\textbf{\Large Re\v{s}enje}\\
\textbf{\Large 37.} U lopti polupre\v{c}nika $a$, u vazduhu, raspodeljeno je naelektrisanje zapreminske gustine $\rho(r) = \rho(a)\frac{r}{a}, 0\leq r\leq a$, gde je $r$ odstojanje posmatrane ta\v{c}ke od sredi\v{s}ta sfere. Koriste\'{c}i se principom superpozicije, odrediti vektor elektri\v{c}nog polja ovog naelektrisanja.
\\\\
\textbf{\Large Re\v{s}enje}\\
\input{solutions/37.tex}
\textbf{\Large 39.} Malo nepokretno telo nalazi se u vakuumu, a njegovo naelektrisanje je $Q = 1\ \mu\mathrm C$. Koliki rad izvr\v{s}i elektri\v{c}na sila pri preme\v{s}tanju drugog malog tela (probnog naelektrisanja), mase $m = 1\ \mu g$ i naelektrisanja $Q_p = 1\ \mathrm{pC}$, (a) iz veoma udaljene ta\v{c}ke u ta\v{c}ku koja je na odstojanju $r_1 = 100\ \mathrm{mm}$ od nepokretnog tela i (b) iz ta\v{c}ke koja je na odstojanju $r_1$ od nepokretnog tela u ta\v{c}ku koja je na odstojanju $r_2 = 1\ \mathrm m$?
\\\\
\textbf{\Large Re\v{s}enje}\\
\textbf{\Large 39.} Malo nepokretno telo nalazi se u vakuumu, a njegovo naelektrisanje je $Q = 1\ \mu\mathrm C$. Koliki rad izvr\v{s}i elektri\v{c}na sila pri preme\v{s}tanju drugog malog tela (probnog naelektrisanja), mase $m = 1\ \mu g$ i naelektrisanja $Q_p = 1\ \mathrm{pC}$, (a) iz veoma udaljene ta\v{c}ke u ta\v{c}ku koja je na odstojanju $r_1 = 100\ \mathrm{mm}$ od nepokretnog tela i (b) iz ta\v{c}ke koja je na odstojanju $r_1$ od nepokretnog tela u ta\v{c}ku koja je na odstojanju $r_2 = 1\ \mathrm m$?
\\\\
\textbf{\Large Re\v{s}enje}\\
\textbf{\Large 39.} Malo nepokretno telo nalazi se u vakuumu, a njegovo naelektrisanje je $Q = 1\ \mu\mathrm C$. Koliki rad izvr\v{s}i elektri\v{c}na sila pri preme\v{s}tanju drugog malog tela (probnog naelektrisanja), mase $m = 1\ \mu g$ i naelektrisanja $Q_p = 1\ \mathrm{pC}$, (a) iz veoma udaljene ta\v{c}ke u ta\v{c}ku koja je na odstojanju $r_1 = 100\ \mathrm{mm}$ od nepokretnog tela i (b) iz ta\v{c}ke koja je na odstojanju $r_1$ od nepokretnog tela u ta\v{c}ku koja je na odstojanju $r_2 = 1\ \mathrm m$?
\\\\
\textbf{\Large Re\v{s}enje}\\
\input{solutions/39.tex}
\textbf{\Large 48.} Osam jednakih ta\v{c}kastih naelektrisanja $Q$ nalaze se u vazduhu, u temenima kocke ivice $a$. Odrediti potencijal ta\v{c}ke $P$ u sredi\v{s}tu kocke ako je referentna ta\v{c}ka (a) u beskona\v{c}nosti ($R_1$) i (b) u sredi\v{s}tu jedne strane ove kocke $(R_2)$
\\\\
\textbf{\Large Re\v{s}enje}\\
\textbf{\Large 48.} Osam jednakih ta\v{c}kastih naelektrisanja $Q$ nalaze se u vazduhu, u temenima kocke ivice $a$. Odrediti potencijal ta\v{c}ke $P$ u sredi\v{s}tu kocke ako je referentna ta\v{c}ka (a) u beskona\v{c}nosti ($R_1$) i (b) u sredi\v{s}tu jedne strane ove kocke $(R_2)$
\\\\
\textbf{\Large Re\v{s}enje}\\
\textbf{\Large 48.} Osam jednakih ta\v{c}kastih naelektrisanja $Q$ nalaze se u vazduhu, u temenima kocke ivice $a$. Odrediti potencijal ta\v{c}ke $P$ u sredi\v{s}tu kocke ako je referentna ta\v{c}ka (a) u beskona\v{c}nosti ($R_1$) i (b) u sredi\v{s}tu jedne strane ove kocke $(R_2)$
\\\\
\textbf{\Large Re\v{s}enje}\\
\input{solutions/48.tex}
\textbf{\Large 51.} Linijsko naelektrisanje kostnatne podu\v{z}ne gustine $Q'$ rasporedjeno je du\v{z} negativnog dela $x$-ose od ta\v{c}ke $(-a,0,0),\ a > 0$ do ta\v{c}ke $(0,0,0)$. Odrediti (a) potencijal ta\v{c}ke $A(a,0,0)$ i (b) potencijal ta\v{c}ke $B(0,a,a)$ uzimaju\'{c}i ta\v{c}ku u beskona\v{c}nosti za referentnu. (v) Koliki je napon izme\dj{}u ta\v{c}aka $A$ i $B$? Sredina je vazduh.
\\\\
\textbf{\Large Re\v{s}enje}\\
\textbf{\Large 51.} Linijsko naelektrisanje kostnatne podu\v{z}ne gustine $Q'$ rasporedjeno je du\v{z} negativnog dela $x$-ose od ta\v{c}ke $(-a,0,0),\ a > 0$ do ta\v{c}ke $(0,0,0)$. Odrediti (a) potencijal ta\v{c}ke $A(a,0,0)$ i (b) potencijal ta\v{c}ke $B(0,a,a)$ uzimaju\'{c}i ta\v{c}ku u beskona\v{c}nosti za referentnu. (v) Koliki je napon izme\dj{}u ta\v{c}aka $A$ i $B$? Sredina je vazduh.
\\\\
\textbf{\Large Re\v{s}enje}\\
\textbf{\Large 51.} Linijsko naelektrisanje kostnatne podu\v{z}ne gustine $Q'$ rasporedjeno je du\v{z} negativnog dela $x$-ose od ta\v{c}ke $(-a,0,0),\ a > 0$ do ta\v{c}ke $(0,0,0)$. Odrediti (a) potencijal ta\v{c}ke $A(a,0,0)$ i (b) potencijal ta\v{c}ke $B(0,a,a)$ uzimaju\'{c}i ta\v{c}ku u beskona\v{c}nosti za referentnu. (v) Koliki je napon izme\dj{}u ta\v{c}aka $A$ i $B$? Sredina je vazduh.
\\\\
\textbf{\Large Re\v{s}enje}\\
\input{solutions/51.tex}

\end{document}
