\clearpage
\textbf{\Large 6.} Dva mala tela nalaze se u vazduhu, na velikom me\dj{}usobnom rastojanju $d$. Tela se naelektri\v{s}u ukupnim naelektrisanjem $Q$ tako da su naelektrisanja tela istog znaka. Pri tome je naelektrisanje jednog tela $Q_1$, a drugog $Q_2 = Q-Q_1$. Odrediti pri kom odnosu naelektrisanja $Q_1 / Q$ \'{c}e elektrostati\v{c}ke sile izme\dj{}u ovih tela biti najja\v{c}e.
\\\\
\textbf{\Large Re\v{s}enje}

\begin{align*}
    |\mathbf{F_{12}}| = |\mathbf{F_{21}}| = \frac{1}{4\pi\varepsilon_0} |Q_1Q_2| 
    = \frac{Q_1(Q-Q_1)}{4\pi\varepsilon_0 d^2}
\end{align*}
\begin{align*}
    |\mathbf{F}| = \frac{1}{4\pi\varepsilon_0d^2}Q\cdot \frac{Q_1}{Q_2}\left(Q - Q\cdot\frac{Q_1}{Q_2}\right) 
    = \frac{Q^2}{4\pi\varepsilon_0d^2}\frac{Q_1}{Q}\left(1 - \frac{Q_1}{Q}\right)
\end{align*}
\begin{align*}
    f\left(\frac{Q_1}{Q}\right) = \frac{Q_1}{Q}\left(1 - \frac{Q_1}{Q}\right),\; 0 \leq \frac{Q_1}{Q} \leq 1
\end{align*}
\begin{align*}
    \boxed{\left(\frac{Q_1}{Q}\right)_\mathrm{opt} = 0,5},\ 
    \boxed{\mathbf{|F|}_\mathrm{max} = \frac{Q^2}{16\pi\varepsilon_0 d^2}}
\end{align*}

\begin{figure}[h]
    \centering
    \begin{tikzpicture}
        \begin{axis}[
            axis y line=center,
            axis x line=middle,
            axis equal,
            %grid=both,
            xmax=1.5,xmin=-0.5,
            ymin=-0.5,ymax=0.5,
            xlabel=$x$,ylabel=$f$,
            %xtick={-10,...,10},
            %ytick={-10,...,10},
            width=15cm,
        ]
        % plot function
        \addplot[samples=150, mark=none, blue] {x * (1 - x)};
        % x mark
        \draw[dashed] (0.5, 0) node [below, outer sep = 5] {$0.5$} -- (0.5, 0.25);
        % y mark
        \draw[dashed] (0, 0.25) node [left, outer sep = 5] {$0.25$} -- (0.5, 0.25);
        \end{axis}
    \end{tikzpicture}
\end{figure}


