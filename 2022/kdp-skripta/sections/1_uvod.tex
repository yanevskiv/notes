\section{Uvod}
\textbf{\v{S}ta su niti?}\\
\textbf{Niti} (eng. \textlat{Threads}) su mehanizam koji omogu\'{c}ava konkurentno ili paralelno izvr\v{s}avanje nekih delova koda programa.\\\\
\textbf{Konkurentno izvr\v{s}avanje} podrazumeva jednu izvr\v{s}nu jedinicu (jezgro ili procesor) koju niti vremenski dele (t.j. izv\v{s}avaju se po principu "\v{c}as jedna, \v{c}as druga, ..."). \\\\
\textbf{Paralelno izvr\v{s}avanje} podrazumeva vi\v{s}e izvr\v{s}nih jedinica (jezgara ili procesora) gde se niti fizi\v{c}ki izvr\v{s}avaju u isto vreme. Na kom jezgru ili procesoru \'{c}e se odredjena nit izvr\v{s}avati i koliko jedinica vremena odlu\v{c}uje \texttt{scheduler} operativnog sistema (\textlat{preemptive multitasking}). Ukoliko ne postoji \texttt{scheduler} ve\'{c} niti same odlu\v{c}uju koja \'{c}e se izvr\v{s}avati slede\'{c}e (\textlat{cooperative multitasking}) takve niti se nazivaju \textbf{korutinama} (eng. \textlat{Coroutines}).

Primer kreiranja dve \textlat{POSIX} niti gde jedna nit ispisuje crveno slovo ``\textlat{\color{red}A}'' dok druga ispisuje zeleno slovo ``\textlat{\color{OliveGreen}B}'':
\begin{lstlisting}
// Compile and run: gcc -pthread main.c && ./a.out
#include <stdio.h>
#include <pthread.h>
#define RED "\x1b[31m"
#define GREEN "\x1b[32m"
#define NOCOLOR "\x1b[0m"

void *a(void *data) {
	for ( ; ; ) {
		printf(RED "A" NOCOLOR);
	}
}
void *b(void *data) {
	for ( ; ; ) {
		printf(GREEN "B" NOCOLOR);
	}
}
int main() {
	pthread_t ta, tb;
	pthread_create(&ta, NULL, a, NULL);
	pthread_create(&tb, NULL, b, NULL);	
	pthread_join(ta, NULL);
	pthread_join(tb, NULL);
	return 0;
}
\end{lstlisting}
\textbf{\v{S}ta je atomi\v{c}nost?}\\
Izraz (t.j. niz operacija) je atomi\v{c}an ukoliko se izvr\v{s}avanje operacija ne mo\v{z}e prekinuti na pola puta t.j. ili se sve operacije izvr\v{s}e odjednom, ili se ne izvr\v{s}i ni jedna.
\\\\
\textbf{\v{S}ta je kriti\v{c}na referenca?}\\
U nekom izrazu koji jedna nit izvr\v{s}ava, \textbf{kriti\v{c}na referenca} je referenca na promenljivu u izrazu \v{c}ija se vrednost menja od strane neke druge niti.
\\\\
\textbf{\v{S}ta je \textlat{At-Most-Once Property (AMOP)}?}\\
Iskaz dodele vrednosti $x=e$ zadovoljava \textbf{\latin At-Most-Once Property} ukoliko je jedna od slede\'{c}e dve stavke ta\v{c}na:
\begin{enumerate}
    \item Izraz $e$ sadr\v{z}i \textbf{najvi\v{s}e jednu} kriti\v{c}nu referencu, i $x$ se ne \v{c}ita od strane drugih niti.
    \item Izraz $e$ ne sadr\v{z}i \textbf{ni jednu} kriti\v{c}nu referencu, a $x$ se mo\v{z}e \v{c}itati od strane drugih niti. 
\end{enumerate}
Ukoliko dodela vrednosti zadovoljava \textlat{AMOP}, u pogledu svih niti \'{c}e izgledati kao da se dodela izvr\v{s}ava atomi\v{c}no.
\\\\
U slede\'{c}em primeru, obe dodele vrednosti $x=e_1$ i $y=e_2$ \textbf{zadovoljavaju} \textlat{AMOP} jer $e_1$ i $e_2$ ne sadr\v{z}e ni jednu kriti\v{c}nu referencu (zadovoljavaju stavku 2):
\begin{lstlisting}
int x = 0, y = 0;
void a() {
    x = x + 1;
}
void b() {
    y = y + 1;
}
\end{lstlisting}
U slede\'{c}em primeru, obe dodele vrednosti $x=e_1$ i $y=e_2$ \textbf{zadovoljavaju} \textlat{AMOP}. U $x=e_1$, $e_1$ sadr\v{z}i kriti\v{c}nu referencu na $y$, ali se $x$ ne \v{c}ita od strane drugih niti (zadovoljava stavku 1). U $x=e_2$, $e_2$ ne sadr\v{z}i ni jednu kriti\v{c}nu referencu (zadovoljava stavku 2).
\begin{lstlisting}
int x = 0, y = 0;
void a() {
    x = y + 1;
}
void b() {
    y = y + 1;
}
\end{lstlisting}
U slede\'{c}em primeru, $x=e_1$ i $y=e_2$ \textbf{ne zadovoljavaju} \textlat{AMOP}. U $x = e_1$, $e_1$ sadr\v{z}i kriti\v{c}nu referencu na $y$, ali $x$ se \v{c}ita od strane niti \texttt{b()} (ne zadovoljava stavku 2.). U $y = e_2$, $e_2$ sadr\v{z}i kriti\v{c}nu referencu na $x$, ali $y$ se \v{c}ita od strane niti \texttt{a()} (ne zadovoljava stavku 2.).
\begin{lstlisting}
int x = 0, y = 0;
void a() {
    x = y + 1;
}
void b() {
    y = x + 1;
}
\end{lstlisting}
\textbf{\v{S}ta je kriti\v{c}na sekcija?}\\
Kriti\v{c}na sekcija je deo koda koji samo jedna nit sme da izvr\v{s}ava u nekom trenutku i mo\v{z}e se implementirati pomo\'{c}u \texttt{mutual exclusion lock}-a (\texttt{mutex}-a).
\\\\
\textbf{\v{S}ta je \textlat{lock}?}\\
Brava (eng. \textlat{Lock}) je mehanizam koji omogu\'{c}ava nekoj niti da ograni\v{c}i pristup nekom delu koda ili resursu ostalim nitima. Na primer, nit mo\v{z}e zaklju\v{c}ati fajl sa eksluzivnim pravom pristupa za upis u fajl (\texttt{exclusive lock}), ili sa deljenim pravom pristupa  za \v{c}itanje iz fajla (\texttt{shared lock}).
\\\\
\textbf{\v{S}ta je \textlat{spinlock}?}\\
\textlat{Spinlock} je brava implementirana pomo\'{c}u uposlenog \v{c}ekanja t.j. petlje koje ne radi ni\v{s}ta.
\begin{lstlisting}
while (lock)
    skip();
\end{lstlisting}
\textbf{\v{S}ta je \textlat{deadlock (livelock)}?}\\
Dve ili vi\v{s}e niti se mogu na\'{c}i u \textlat{deadlock}-u ukoliko blokiraju jedna druge tako da ni jedna ne mo\v{z}e da nastavi sa izvr\v{s}avanjem. \textlat{Livelock} je sli\v{c}an \textlat{deadlock}-u s tim \v{s}to niti stalno menjaju ustupaju prednost nekoj drugoj niti i na taj na\v{c}in tako\dj{}e ni jedna nit ne nastavlja sa izvr\v{s}avanjem.
\\\\
\textbf{\v{S}ta je izgladnjivanje (\textlat{starvation})?}\\
Izgladnjivanje niti se javlja kada nit predugo \v{c}eka na nekom uslovu. Na primer, kod \textlat{Readers/Writers} problema se mo\v{z}e javiti izgladnjivanje pisaca ukoliko \v{c}itaoci stalno nadolaze.
\\\\
\textbf{\v{S}ta je \texttt{volatile}?}\\
\texttt{volatile} je klju\v{c}na re\v{c} u jezicima \textlat{C} i \textlat{C++} koja nagove\v{s}tava prevodiocu da ne pravi optimizacije sa objektnom nazna\v{c}enom sa \texttt{volatile}.
Na primer, ukoliko \v{z}elimo da radimo uposleno \v{c}ekanje tako \v{s}to \'{c}emo imati jednu \texttt{for} petlju koja ne radi ni\v{s}ta:
\begin{lstlisting}
// gcc -masm=intel -S -O2 main.c && cat ./main.s
int main() {
    /* volatile */ int i;
    for (i = 0; i < 100000; i++);
    return 0;
}
\end{lstlisting}
Prevodilac \'{c}e optimizovati kod tako \v{s}to uop\v{s}te ne\'{c}e generisati kod za petlju:
\begin{lstlisting}[language=]
        .globl  main
        .type   main, @function
main:
        xor     eax, eax
        ret
\end{lstlisting}
Dok ukoliko otkomentari\v{s}emo \texttt{volatile}, kod za petlju se generi\v{s}e:
\begin{lstlisting}[language=]
        .globl  main
        .type   main, @function
main:
        mov     DWORD PTR -4[rsp], 0
        mov     eax, DWORD PTR -4[rsp]
        cmp     eax, 99999
        jg      .L2
        .p2align 4,,10
        .p2align 3
.L3:
        mov     eax, DWORD PTR -4[rsp]
        add     eax, 1
        mov     DWORD PTR -4[rsp], eax
        mov     eax, DWORD PTR -4[rsp]
        cmp     eax, 99999
        jle     .L3
.L2:
        xor     eax, eax
        ret
\end{lstlisting}
Pravilo je da sve deljene promenljive treba ozna\v{c}iti sa modifikatorom \texttt{volatile}. 
\\\\
\textbf{Napomena:} U jeziku \textlat{C++} \textbf{nije dovoljno} koristiti \texttt{volatile} za korektnu sinhronizaciju pomo\'{c}u \textlat{spinlock}-a ve\'{c} za implementaciju neophodno koristiti konstrukte iz zaglavlja \texttt{<atomic>}, tako da sve kodove navedene u nastavku treba razumeti samo kao pseudo kod.
\\\\
\textbf{\v{S}ta je \texttt{skip()}?}\\
Pozviom \texttt{skip()}, nit se odri\v{c}e procesorskog vremena. Radi optimizacije, \texttt{skip()} je potrebno pozvati prilikom uposlenog \v{c}ekanja ukoliko uslov nije ispunjen. 
\\\\
Na primer, ukoliko je \texttt{scheduler} dodelio procesor nekoj niti u vremenu od $150\ \mathrm{ms}$, a ona ustanovi u roku od $7\ \mu s$ da uslov za nastavak nije ispunjen, nema potrebe da se nit vrti u petlji narednih $149,993\ \mathrm{ms}$ ukoliko postoj neka druga nit koja bi se mogla izvr\v{s}avati.
\\\\
\texttt{skip()} je mogu\'{c}e implementirati kao \texttt{sched\_yield()} ili \texttt{usleep(0)}.