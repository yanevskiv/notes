\clearpage
\section{\latin Dining philosophers}
\textbf{\large Zadatak} Proces je pokrenuo 5 niti koje simuliraju pona\v{s}anje filozofa za stolom. Filozof neko vreme misli, uzima levu i desnu vilju\v{s}ku na stolu, jede ru\v{c}ak na tanjiru, vra\'{c}a vilju\v{s}ke, i nastavlja da misli. Filozof ne mo\v{z}e da uzme levu vilju\v{s}ku ukoliko ju je trenutno filozof levo koristi, i sli\v{c}no ne mo{z}e da uzme desnu vilju\v{s}ku ukoliko ju je trenutno filozof desno koristi. Ukoliko svih pet filozofa u isto vreme poku\v{s}aju da uzmu levu vilju\v{s}ku, ni jedan filozof ne\'{c}e mo\'{c}i da uzme desnu vilju\v{s}ku, i tada nastupa deadlock. Implementirati pona\v{s}anje filozofa pomo\'{c}u monitora tako da se ovaj deadlock ne mo\v{z}e dogoditi.
\\\\
\textbf{\large 1. na\v{c}in}\\
Re\v{s}enje gde obr\'{c}emo uzimanja vilju\v{s}aka redosled parnim filozofa.
\\\\
\textbf{Re\v{s}enje sa \texttt{Signal-and-Wait} disciplinom}
\begin{lstlisting}
#include <thread.h>
#include <monitor.h>
#define N 5
#define LEFT(id) ((id) == 0 ? N - 1 : (id) - 1)
#define RIGHT(id) (((id) + 1) % N)

class Table : SW_Monitor {
    bool m_forkFree[N] = { true, true, true, true, true };
    cond m_waitForFork[N];
public:
    Table() : m_waitForFork {this, this, this, this, this} {
        
    }
    void acquire_fork(int which) {
        monitor {
            if (! m_forkFree[which]) 
                m_waitForFork[which].wait();
            m_forkFree[which] = false;
        }
    }

    void release_fork(int which) {
        monitor {
            m_forkFree[which] = true;
            m_waitForFork[which].signal();
        }
    }
} table;

void phil(int id) {
    int left = LEFT(id);
    int right = RIGHT(id);
    while (true) {
        // Think
        think();
        // Acquire forks
        if (id % 2 == 0) {
            table.acquire_fork(left);
            table.acquire_fork(right);
        } else {
            table.acquire_fork(right);
            table.acquire_fork(left);
        }
        // Eat
        eat();
        // Release forks
        table.release_fork(left);
        table.release_fork(right);
    }
}

int main() {
    Thread tp[N];
    for (int i = 0; i < N; i++) 
        tp[i] = createThread(phil, i);
    for (int i = 0; i < N; i++) 
        join(tp[i]);
    return 0;
}

\end{lstlisting}
\textbf{Re\v{s}enje sa \texttt{Signal-and-Continue} disciplinom}
\begin{lstlisting}
#include <thread.h>
#include <monitor.h>
#define N 5
#define LEFT(id) ((id) == 0 ? N - 1 : (id) - 1)
#define RIGHT(id) (((id) + 1) % N)

class Table : SC_Monitor {
    bool m_forkFree[N] = { true, true, true, true, true };
    bool m_needFork[N] = { false, false, false, false, false};
    int  m_ticket = N - 1;
    cond m_waitForFork[N];
    cond m_waitForTicket;
public:
    Table() 
        : m_waitForFork {this, this, this, this, this}
        , m_waitForTicket {this} {
        
    }
    void acquire_fork(int which) {
        monitor {
            if (! m_forkFree[which]) {
                m_needFork[which] = true;
                m_waitForFork[which].wait();
            } else {
                m_forkFree[which] = false;
            }
        }
    }
    void release_fork(int which) {
        monitor {
            m_forkFree[which] = true;
            if (m_needFork[which]) {
                m_forkFree[which] = false;
                m_needFork[which] = false;
                m_waitForFork[which].signal();
            }
        }
    }
    void acquire_ticket() {
        monitor {
            while (m_ticket == 0)
                m_waitForTicket.wait();
            m_ticket -= 1;
        }
    }
    void release_ticket() {
        monitor {
            m_ticket += 1;
            m_waitForTicket.signal();
        }
    }
} table;

void phil(int id) {
    int left = LEFT(id);
    int right = RIGHT(id);
    while (true) {
        // Think
        think();
        // Acquire forks
        table.acquire_ticket();
        table.acquire_fork(left);
        table.acquire_fork(right);
        // Eat
        eat();
        // Release forks
        table.release_fork(left);
        table.release_fork(right);
        table.release_ticket();
    }
}

int main() {
    Thread tp[N];
    for (int i = 0; i < N; i++) 
        tp[i] = createThread(phil, i);
    for (int i = 0; i < N; i++) 
        join(tp[i]);
    return 0;
}

\end{lstlisting}
\clearpage
\textbf{\large 2. na\v{c}in}\\
Re\v{s}enje gde imamo ``\textlat{ticket}'' inicijalizovan na $N-1$, gde je $N$ broj filozofa.
\\\\
\textbf{Re\v{s}enje sa \texttt{Signal-and-Wait} disciplinom}
\begin{lstlisting}
#include <thread.h>
#include <monitor.h>
#define N 5
#define LEFT(id) ((id) == 0 ? N - 1 : (id) - 1)
#define RIGHT(id) (((id) + 1) % N)

class Table : SW_Monitor {
    bool m_forkFree[N] = { true, true, true, true, true };
    bool m_needFork[N] = { false, false, false, false, false};
    int  m_ticket = N - 1;
    cond m_waitForFork[N];
    cond m_waitForTicket;
public:
    Table() 
        : m_waitForFork {this, this, this, this, this}
        , m_waitForTicket {this} {
        
    }
    void acquire_fork(int which) {
        monitor {
            if (! m_forkFree[which])
                m_waitForFork[which].wait();
            m_forkFree[which] = false;
        }
    }
    void release_fork(int which) {
        monitor {
            m_forkFree[which] = true;
            m_waitForFork[which].signal();
        }
    }
    void acquire_ticket() {
        monitor {
            if (m_ticket == 0)
                m_waitForTicket.wait();
            m_ticket -= 1;
        }
    }
    void release_ticket() {
        monitor {
            m_ticket += 1;
            m_waitForTicket.signal();
        }
    }
} table;

void phil(int id) {
    int left = LEFT(id);
    int right = RIGHT(id);
    while (true) {
        // Think
        think();
        // Acquire forks
        table.acquire_ticket();
        table.acquire_fork(left);
        table.acquire_fork(right);
        // Eat
        eat();
        // Release forks
        table.release_fork(left);
        table.release_fork(right);
        table.release_ticket();
    }
}

int main() {
    Thread tp[N];
    for (int i = 0; i < N; i++) 
        tp[i] = createThread(phil, i);
    for (int i = 0; i < N; i++) 
        join(tp[i]);
    return 0;
}

\end{lstlisting}
\textbf{Re\v{s}enje sa \texttt{Signal-and-Continue} disciplinom}
\begin{lstlisting}
#include <thread.h>
#include <monitor.h>
#define N 5
#define LEFT(id) ((id) == 0 ? N - 1 : (id) - 1)
#define RIGHT(id) (((id) + 1) % N)

class Table : SC_Monitor {
    bool m_forkFree[N] = { true, true, true, true, true };
    bool m_needFork[N] = { false, false, false, false, false};
    int  m_ticket = N - 1;
    cond m_waitForFork[N];
    cond m_waitForTicket;
public:
    Table() 
        : m_waitForFork {this, this, this, this, this}
        , m_waitForTicket {this} {
        
    }
    void acquire_fork(int which) {
        monitor {
            if (! m_forkFree[which]) {
                m_needFork[which] = true;
                m_waitForFork[which].wait();
            } else {
                m_forkFree[which] = false;
            }
        }
    }
    void release_fork(int which) {
        monitor {
            m_forkFree[which] = true;
            if (m_needFork[which]) {
                m_forkFree[which] = false;
                m_needFork[which] = false;
                m_waitForFork[which].signal();
            }
        }
    }
    void acquire_ticket() {
        monitor {
            while (m_ticket == 0)
                m_waitForTicket.wait();
            m_ticket -= 1;
        }
    }
    void release_ticket() {
        monitor {
            m_ticket += 1;
            m_waitForTicket.signal();
        }
    }
} table;

void phil(int id) {
    int left = LEFT(id);
    int right = RIGHT(id);
    while (true) {
        // Think
        think();
        // Acquire forks
        table.acquire_ticket();
        table.acquire_fork(left);
        table.acquire_fork(right);
        // Eat
        eat();
        // Release forks
        table.release_fork(left);
        table.release_fork(right);
        table.release_ticket();
    }
}

int main() {
    Thread tp[N];
    for (int i = 0; i < N; i++) 
        tp[i] = createThread(phil, i);
    for (int i = 0; i < N; i++) 
        join(tp[i]);
    return 0;
}

\end{lstlisting}
\clearpage
\textbf{\large 3. na\v{c}in}\\
Re\v{s}enje gde uzimamo obe vilju\v{s}ke u isto vreme i vra\'{c}amo obe vilju\v{s}ke u isto vreme, pritom bude\'{c}i filozofa levo ili desno ukoliko oni ranije nisu uspeli.
\\\\
\textbf{Re\v{s}enje sa \texttt{Signal-and-Wait} disciplinom}
\begin{lstlisting}
#include <thread.h>
#include <monitor.h>
#define N 5
#define LEFT(id) ((id) == 0 ? N - 1 : (id) - 1)
#define RIGHT(id) (((id) + 1) % N)

class Table : SW_Monitor {
    bool m_forkFree[N] = { true, true, true, true, true };
    bool m_hungry[N] = { false, false, false, false, false };
    cond m_waitForForks[N];
public:
    Table() : m_waitForForks {this, this, this, this, this} {
        
    }
    void acquire_forks(int left, int right) {
        monitor {
            int id = RIGHT(left); // same as LEFT(right)
            if (! (m_forkFree[left] && m_forkFree[right])) {
                m_hungry[id] = true;
                m_waitForForks[id].wait();
            }
            m_hungry[id] = false;
            m_forkFree[left] = false;
            m_forkFree[right] = false;
        }
    }

    void release_forks(int left, int right) {
        monitor {
            m_forkFree[left] = true;
            m_forkFree[right] = true;
            if (m_hungry[left] && m_forkFree[LEFT(left)])  {
                m_waitForForks[left].signal();
            } else if (m_hungry[right] && m_forkFree[RIGHT(right)]) {
                m_waitForForks[right].signal();
            }
        }
    }
} table;

void phil(int id) {
    int left = LEFT(id);
    int right = RIGHT(id);
    while (true) {
        // Think
        think();
        // Acquire forks
        table.acquire_forks(left, right);
        // Eat
        eat();
        // Release forks
        table.release_forks(left, right);
    }
}

int main() {
    Thread tp[N];
    for (int i = 0; i < N; i++) 
        tp[i] = createThread(phil, i);
    for (int i = 0; i < N; i++) 
        join(tp[i]);
    return 0;
}

\end{lstlisting}
\textbf{Re\v{s}enje sa \texttt{Signal-and-Continue} disciplinom}
\begin{lstlisting}
#include <thread.h>
#include <monitor.h>
#define N 5
#define LEFT(id) ((id) == 0 ? N - 1 : (id) - 1)
#define RIGHT(id) (((id) + 1) % N)

class Table : SC_Monitor {
    bool m_forkFree[N] = { true, true, true, true, true };
    bool m_hungry[N] = { false, false, false, false, false };
    cond m_waitForForks[N];
public:
    Table() : m_waitForForks {this, this, this, this, this} {
        
    }
    void acquire_forks(int left, int right) {
        monitor {
            int id = RIGHT(left); // same as LEFT(right)
            if (! (m_forkFree[left] && m_forkFree[right])) {
                m_hungry[id] = true;
                m_waitForForks[id].wait();
            } else {
                m_forkFree[left] = false;
                m_forkFree[right] = false;
            }
        }
    }

    void release_forks(int left, int right) {
        monitor {
            m_forkFree[left] = true;
            m_forkFree[right] = true;
            if (m_hungry[left] && m_forkFree[LEFT(left)])  {
                m_hungry[left] = false;
                m_forkFree[left] = false;
                m_forkFree[LEFT(left)] = false;
                m_waitForForks[left].signal();
            } else if (m_hungry[right] && m_forkFree[RIGHT(right)]) {
                m_hungry[right] = false;
                m_forkFree[right] = false;
                m_forkFree[RIGHT(right)] = false;
                m_waitForForks[right].signal();
            }
        }
    }
} table;

void phil(int id) {
    int left = LEFT(id);
    int right = RIGHT(id);
    while (true) {
        // Think
        think();
        // Acquire forks
        table.acquire_forks(left, right);
        // Eat
        eat();
        // Release forks
        table.release_forks(left, right);
    }
}

int main() {
    Thread tp[N];
    for (int i = 0; i < N; i++) 
        tp[i] = createThread(phil, i);
    for (int i = 0; i < N; i++) 
        join(tp[i]);
    return 0;
}

\end{lstlisting}
