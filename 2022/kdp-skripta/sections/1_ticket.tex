\clearpage
\section{\textlat{Ticket} algoritam}
Kod \textbf{\textlat{Ticket}} algoritma, nit koja \v{z}eli da udje u kriti\v{c}nu sekciju najpre atomi\v{c}no dohvata globalni broj \texttt{ticket} u lokalnu promenljivu \texttt{number} i inkrementira globalni \texttt{ticket} za vrednost 1. Globalni \texttt{ticket} se atomi\v{c}no dohvata i inkrementira pozivom specijalne instrukcije \texttt{fetch\_and\_add(x, v)}. Nit potom \v{c}eka dok globalna vrednost \texttt{next} (``slede\'{c}i'') ne postane ba\v{s} vrednost prethodno dohvatanog broja \texttt{number}. Kada nit zavr\v{s}i sa kriti\v{c}nom sekcijom, slede\'{c}oj niti se dopu\v{s}ta ulazak u kriti\v{c}nu sekciju sa \texttt{next += 1} (nije neophodno da se vrednost \texttt{next} inkrementira specijalnom instrukcijom \texttt{fetch\_and\_add(x, v)} jer se kriti\v{c}na sekcija \texttt{/*CRITICAL SECTION*/} nastavlja i na sekciju \texttt{/*UNLOCK*/}).  Ticket algoritam garantuje ``fer'' (\textlat{FIFO}) ulazak u kriti\v{c}nu sekciju.
\begin{lstlisting}
volatile int ticket = 0;
volatile int next = 0;

void thread(int id)
{
    while (true) {
		/* LOCK */
        int number = fetch_and_add(&ticket, 1);
        while (next != number)
            skip();
		/* CRITICAL SECTION */
		work();
		/* UNLOCK */
		next += 1;
		/* NON-CRITICAL SECTION */
        usleep(50);
    }
}
\end{lstlisting}