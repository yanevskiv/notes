\clearpage
\section{\latin Simple synchronization}
\textbf{\large Zadatak} Proces je pokrenuo dve niti sa beskona\v{c}nim petljama, gde prva nit ispisuje na konzoulu znak \textlat{``a''} dok druga nit ispisuje znak ``\textlat{b}'' (\textlat{aaaaaaaaabbbbbbbbbbbaaaaaaaabbbb...}). Koriste\'{c}i semafore, potrebno je sinhronizovati niti tako da se na konzoli naizmeni\v{c}no ispisuju znakovi \textlat{``a''} i \textlat{``b''}. (\textlat{abababababababab...}).
\\\\
\textbf{\v{S}ta je raspodeljeni binarni semafor?}\\
Raspodeljeni binarni semafor je binarni semafor (semafor \v{c}ija se vrednost kre\'{c}e izmedju 0 ili 1) \v{c}iji su pozivi operacije \texttt{wait(s)} raspodeljeni unutar koda jedne niti, a pozivi operacije \texttt{signal(s)} raspodeljeni unutar koda druge niti. Ovime jedna nit mo\v{z}e da kontroli\v{s}e tok druge niti. Naime, nit koja poziva \texttt{signal(s)} kontroli\v{s}e izvr\v{s}avanje niti koja poziva \texttt{wait(s)}.
\\\\
\textbf{\large Re\v{s}enje:} 
\begin{lstlisting}
#include <thread.h>
#include <sem.h>

Sem sa = 1, sb = 0; // dva raspodeljena binarna semafora

void a() 
{
    while (true) {
        wait(sa);
        print("a");
		signal(sb);
    }
}

void b() 
{
    while (true) {
        wait(sb);
        print("b");
		signal(sa);
    }
}


int main() {
    Thread ta = createThread(a);
    Thread tb = createThread(b);
    join(ta);
    join(tb);
    return 0;
}

\end{lstlisting}