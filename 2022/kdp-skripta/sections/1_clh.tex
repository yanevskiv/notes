\clearpage
\section{\textlat{CLH} algoritam}
Kod \textbf{\textlat{CLH (Craig, Landin, and Hagerste)}} algoritma postoji globalna brava postoji globalna brava \texttt{clh\_lock} koja je inicijalno kreirana kao otklju\v{c}ana. Nit koja \v{z}eli da udje u kriti\v{c}nu sekciju kreira novu \textbf{zaklju\v{c}anu} bravu i atomi\v{c}no je zamenjuje sa globalnom bravom \texttt{clh\_lock} dohvataju\'{c}i staru vrednost globalne brave u \texttt{old\_lock}. Stara vrednost globalne brave se dohvata i postavlja na novu vrednost pozivom specijalne instrukcije \texttt{get\_and\_set(x, v)}. Nit potom \v{c}eka dok stara brava (koju je neko drugi naparvio i treba da otklju\v{c}a) ne postane otklju\v{c}ana, ulazi u kriti\v{c}nu sekciju, i potom otklju\v{c}ava bravu \texttt{new\_lock}, propu\v{s}taju\'{c}i nit \v{c}ija je to brava \texttt{old\_lock}. Prva nit koja naidje \'{c}e u\'{c}i u kriti\v{c}nu sekciju dok se ostale niti blokiraju na poslednju bravu koja je zamenjena. Ovim se niti virtuelno ulan\v{c}avaju i propu\v{s}taju po onom redosledu kom su pristigle. Kao i \textlat{Ticket} i Andersonov algoritam, \textlat{CLH} algoritam garantuje ``fer'' (\textlat{FIFO}) redosled ulaska u kriti\v{c}nu sekciju.

\begin{lstlisting}
typedef bool Lock;

volatile Lock *clh_lock = new Lock { false };

void thread(int id)
{
    while (true) {
        /* LOCK */
        volatile Lock *new_lock = new Lock { true };
        volatile Lock *old_lock = get_and_set(&clh_lock, new_lock);
        while (*old_lock == true)
            skip();
        delete old_lock;
        /* CRITICAL SECTION */
        work();
        /* UNLOCK */
        *new_lock = false;
        /* NON-CRITICAL SECTION */
        usleep(50);
    }
}
\end{lstlisting}