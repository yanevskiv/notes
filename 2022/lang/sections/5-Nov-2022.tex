\section{5-Nov-2022 (Sat)}
\subsection{くらいのものだ (kurai no mono da): only}

わたしのほうでできることは、あなたと連絡を保つことくらいのものです。
僕が買える家は、これぐらいのものだ。
「承知しましたとも。それだけですか?」「あとはここでいっしょに食事していただくことくらいのものです。」
彼が仕事を辞めないように説得できるのは、あなたぐらいのものです。
人間はだいたい侮辱されるのが大好きなものですが、お気づきですか?これがあればこそどうにか生きていけると言ってもいいくらいのものです。
全部でせいぜい五千円か六千円くらいのものだが、僕にとっちゃ大事なものなんだ。
この部類の大多数は、悲劇には会わないでしまう。せいぜい、死んでもいい年ごろになって、多少とも肝臓を悪くするくらいのものである。
最近二年の間、彼は家庭内の問題については、ほんのだいたいのことを、聞きかじりに知っているくらいのもので、それ以上詳しく立ち入ることをよしてしまっていた。
僕はけっして何百万なんて金を譲り受けはしませんでしたよ。おそらく僕の持ってるのは、あんたが予想されたののせいぜい十分の一くらいのものでしょう。
アメリカにいたころ、子供同士の友達だったんです。こっちへ来てから、一二度出会ったくらいのもので、しみじみと会って、話したということはなかったです。
あれは、よくあんなことがありますよ、しかし、昨日みたいに激しいのは、まあ三年に一度くらいのもので、けっしてそれより多いことはありません!
それにどこへ逃げたらいいですかね!外国ですか?外国へなんか逃亡するのはポーランド人くらいのもので、その男じゃありませんよ、ましてこっちは監視しているんだし、手も打ってあるんですからな。
Самое большее, что я могу сделать со своей стороны, это поддерживать с вами контакт.
Это единственный дом, который я могу себе позволить.
Да, сэр. И это все? Все, что осталось - это поесть с нами здесь.
Вы единственный, кто может убедить его не бросать работу.
Замечали ли вы, что люди вообще любят, когда их оскорбляют? Это гораздно больше, чем мы можем сказать, что это единственный способ жить.
В целом, это примерно 5000 или 6000 иен максимум, но для меня это важно.
Большинство этой части населения не встретится с трагедией. В лучшем случае, они достаточно стары для смерти и имеют некоторые поверждения печени.
Я знал об этом достаточно, чтобы услышать об этом, и запретил себе вдаваться в подробности.
Я бы никогда не отдал миллионы долларов. Вероятно, у меня есть не более десятой части того, что вы ожидали.
Когда я был в США, мы дружили с другими детьми. После приезда сюда я встретил их всего один или два раза, и мы никогда по-настоящему не встречались и не разговаривали друг с другом.
Такое случается часто, но что-то такое жестокое, как вчера, ну, это случается только раз в три года, не чаще!
Кроме того, куда бы я побежала! За границей? Это только поляки бегут в чужие страны, а не этот человек, и мы следим за вами и приняли меры.

\subsection{くらいなら (kurai nara) rather than…}

彼の信頼を裏切るくらいなら、むしろ死を選ぶ。
どうせ旅籠屋で泊るくらいなら、自分の家で一晩過すのもよかろうと思った。
この秘密があばかれて、子供たちを汚らわしい家名に苦しめるくらいなら、いっそわたしが死刑になってもよいと思ったのです。
あの人と結婚できないくらいなら死んじまうわ。
父は、おれを自分のような貿易商にしたいんだ、ところがおれはそんなものになるくらいなら、弾に撃たれて死んだほうがいい。
こんなところへ来るくらいなら、戦場にいたほうが遥かに安全でしたよ。
おれが二度とお前を横目でだって見るくらいなら、からだも魂もとっくに地獄に行っちまう方がましだ。
他の人たちに愛されないくらいなら、私、生きるよりも死ぬほうをとるわ。
山田春美を解任するくらいなら、その前に自分の解任を要求しますよ。
わたしたちはいつになったって、娘を嫁にやることはできやしません。それくらいなら、いっそ田舎へひっこんでしまうほうがましです。
小銭くらいならいつでも稼ぎだせる腕も持っているのだし、外遊の金ぐらいいまだってちゃんと用意しているのだから。
フロックコートをぬぐくらいならまだしも、下着までぬいでくれと頼むのである。いや、頼んだのではない、実際は命令したのだ。彼にはそれがよくわかった。
ところが、あなたはいまも、告白するくらいなら徒刑にでも行ったほうがましだなんて叫んでおられるんですからねえ。
ロバート将軍は、至急報に自らの名前が目立つくらいならポケットに生きたガラガラヘビを入れてもらったほうがましだと、部下の広報担当仕官に語ったことがある。
Я скорее умру, чем предам его доверие.
Я подумал, что было бы неплохо провести ночь в собственном доме, а не останавливатся в гостнице.
Я бы предпочел быть преданным смерти, чем чтобы эта тайна была раскрыта и чтобы мои дети страдали под грязной фамилией.
Я скорее умру, чем выйду замуж за этого человека.
Мой отец хочет, чтобы я стал торговцем, как он, но я скорее умру от пули, чем стану им.
Я был бы в большей безопасности на поле боя, чем в таком месте, как это.
Я бы предпочла, чтобы мое тело и душа давно отправились в ад, чем еще хоть раз взглянуть на тебя косо.
Я лучше умру, чем буду жить, чем не буду любим другими.
Если бы я предпочел уволить Харуми Ямаду, я бы потребовал перед этим своего уволенения.
Мы никогда не сможем выдать нашу дочь замуж. Я бы предпочел уехать на пенсию в деревню.
Он умеет зарабатывать мелочь в любое время, и у него остается достаточно денег на прогулки.
Он просил ее раздеться до нижего белья, хотя бы для того, чтобы снят фрак. Нет, он не спрашивал, он приказал. Он точно знал, что делает.
Но даже сейчас вы кричите, что скорее сядете в тюрьму, чем признаетесь.
Генерал Роберт однажды сказал подчинненому офицеру по связям с общественностью, что он предпочел бы иметь в кармане живую гремучую змею, чем чтобы его имя красовалось на видном месте в срочном докладе.