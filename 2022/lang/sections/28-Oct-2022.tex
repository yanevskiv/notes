\section{28-Oct-2022 (Fri)}
\subsection{嫌いがある (kirai ga aru) to have a tendency to}

人はとかく自身に都合がいい意見にのみ耳を傾けるきらいがある。
彼はどうも物事を悲観的に考えるきらいがある。
父は自分と違う考え方を認めようとしないきらいがある。
人は年をとると、周りの人の忠告に耳を貸さなくなるきらいがある。
彼女は、何でもものごとを悪い方に考えるきらいがある。
あの人はものごとを大げさに言うきらいがある。
彼は物事を少し考えすぎるきらいがある。
インドではまだ女性を低く見るきらいがある。
白鳥はおなじ卓のむかいに座を占めて、熱心にじれったそうに用件の説明に耳をかたむけながら、ひっきりなしにかわるがわる二人に目を移していたが、それがやや度を越しているきらいがあった。